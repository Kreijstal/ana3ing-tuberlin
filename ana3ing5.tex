\documentclass{scrartcl}
\def\wochenumber{5}
\usepackage{amsmath}
\usepackage{amsfonts}
\usepackage{tabto}
\usepackage{enumerate}
\usepackage[ngerman]{babel}
%\usepackage[T1]{fontenc}
\usepackage{lmodern}
\usepackage[utf8]{inputenc}
\usepackage{tabularx}
\usepackage[paper=a4paper,left=25mm,right=25mm,top=20mm,bottom=20mm]{geometry}
\usepackage{nccmath}
\usepackage{xcolor}
\usepackage{ragged2e}
\usepackage{graphics}
\usepackage{graphicx}
\usepackage{fancyhdr}
\usepackage{mathtools}  
\usepackage{float}
\usepackage{physics}
\usepackage{subcaption}
%\usepackage{wasysym}
%\newcommand*{\vpointer}{\vcenter{\hbox{\scalebox{2}{\Huge\pointer}}}}
\usepackage[linkcolor=blu,colorlinks=false]{hyperref}

%\usepackage{cfr-lm} %ich erinnere mich nicht wieso war das wichtig
\usepackage{newtxmath}
\usepackage{cleveref}
\renewcommand{\headrulewidth}{0.5pt}
\renewcommand{\footrulewidth}{0.5pt}
\usepackage{microtype}  % Minature font tweaks
\usepackage{cancel}
%\usepackage{cmathbb}
\usepackage{tikz,calc}
\usepackage{tkz-fct}  
\usepackage{tkz-euclide}
\usepackage{derivative}
\usepackage{ifthen}
\usetikzlibrary{decorations.pathreplacing,decorations.markings}
%\usetikzlibrary{decorations.markings}
\tikzset{
    set arrow inside/.code={\pgfqkeys{/tikz/arrow inside}{#1}},
    set arrow inside={end/.initial=>, opt/.initial=},
    /pgf/decoration/Mark/.style={
        mark/.expanded=at position #1 with
        {
            \noexpand\arrow[\pgfkeysvalueof{/tikz/arrow inside/opt}]{\pgfkeysvalueof{/tikz/arrow inside/end}}
        }
    },
    arrow inside/.style 2 args={
        set arrow inside={#1},
        postaction={
            decorate,decoration={
                markings,Mark/.list={#2}
            }
        }
    },
}
\tikzset{
  % style to apply some styles to each segment of a path
  on each segment/.style={
    decorate,
    decoration={
      show path construction,
      moveto code={},
      lineto code={
        \path [#1]
        (\tikzinputsegmentfirst) -- (\tikzinputsegmentlast);
      },
      curveto code={
        \path [#1] (\tikzinputsegmentfirst)
        .. controls
        (\tikzinputsegmentsupporta) and (\tikzinputsegmentsupportb)
        ..
        (\tikzinputsegmentlast);
      },
      closepath code={
        \path [#1]
        (\tikzinputsegmentfirst) -- (\tikzinputsegmentlast);
      },
    },
  },
  % style to add an arrow in the middle of a path
  mid arrow/.style={postaction={decorate,decoration={
        markings,
        mark=at position .5 with {\arrow[#1]{stealth}}
      }}},
}

\usepackage[german=quotes]{csquotes}
%\DeclareSymbolFont{eulerletters}{U}{zeur}{b}{n}
%\DeclareSymbolFont{cmrot1italics}{OT1}{cmr}{m}{it}
%\DeclareMathSymbol{\imag}{\mathord}{eulerletters}{`i}
%\DeclareMathSymbol{\imag}{\mathord}{cmrot1italics}{`i}
\DeclareSymbolFont{uiletters}{OT1}{cmr}{m}{ui}
\DeclareMathSymbol{\imag}{\mathalpha}{uiletters}{`i}
% Common shortcuts
\def\mbb#1{\mathbb{#1}}
\def\mfk#1{\mathfrak{#1}}
%\def\imag{\mathsf{i}}
\def\bN{\mbb{N}}
\def\bC{\mbb{C}}
\def\bR{\mbb{R}}
\def\bQ{\mbb{Q}}
\def\bZ{\mbb{Z}}
\newcommand{\func}[3]{#1\colon#2\to#3}
\newcommand{\vfunc}[5]{\func{#1}{#2}{#3},\quad#4\longmapsto#5}
\newcommand{\floor}[1]{\left\lfloor#1\right\rfloor}
\newcommand{\ceil}[1]{\left\lceil#1\right\rceil}
%Abstand zwischen Absätzen, Zeilenabstände
\voffset26pt 
\parskip6pt
%\parindent1cm  %Rückt erste Zeile eines neuen Absatzes ein
\usepackage{setspace}
\onehalfspacing
\title{Analysis III für Ingenieurwissenschaften}
\author{Juan Pardo Martin (397882)\and Tuan Kiet Nguyen (404029)\and Leonardo Nerini (414193)}
\let\*\cdot
\let\-\rightarrow
\let\_\Rightarrow
\let\>\Leftrightarrow
\definecolor{l}{rgb}{0.0, 0.5, 0.0}
\newcommand\ReIm[0]{\text{ReIm}}
\makeatletter
\newcommand\sqrtnorm[1]{
    \count@=0
    \sqrt{
    \@for\levar:=#1\do{
        \advance\count@ 1
        \ifnum\count@=1
        \else +
        \fi
        {\levar}^2
        }
    }
}
\makeatother
\let\c\textcolor
%\mathbb{R} 
%\begin{align*}
\titlehead
{
\begin{tabular}{ll}
\begin{minipage}{0.5\textwidth}
%\begin{figure}[H]
% \raggedright
 %\includegraphics[scale=0.04]{tu-logo}\\
%\end{figure}
\end{minipage}
\begin{minipage}{0.5\textwidth}
\begin{figure}[H]
\raggedleft
%\includegraphics[scale=0.04]{tu-logo}\\
\end{figure}
\end{minipage}
\end{tabular}\\
\\
  \small
      {
    Ana III Hausaufgabe, \wochenumber{} Woche\\
    Tutor:  David Sering  \\
	SS 2021
    }


}
%\usepackage{mathtools} 
\begin{document}
\maketitle
%\author
\begin{section}{Aufgabe}%Aufgabe 1
    (Hausaufgabe zum Stoff der Ubung 4)\\
    Gegeben seien folgende zwei Kurven:
    \begin{align*}
        \gamma_1&: t\mapsto e^{\imag t},\quad \frac{\pi}{2}\leq t \leq \pi,\\
        \gamma_2&: t\mapsto 1+\imag t,\quad -1\leq t \leq 2.
    \end{align*}
Berechnen Sie die folgenden komplexen Kurvenintegrale mit der Definition als Parameterintegrale:
\[\text{a)} \int_{\gamma_1}\frac{\log(\imag \log \overline z)}{z}dz,\text{ b) }\int_{\gamma_2}\frac{1}{z^2}dz\]

\begin{subsection}{Antwort}
    %der Zusammenhang zwischen Kurvenintegral und Vektorfeldern: 
    %\[\int_C f(z)dz=\int_C \begin{psmallmatrix}
    %    u\\-v
    %\end{psmallmatrix}\*d\vec{s}+\imag \int_C \begin{psmallmatrix}
    %    v\\u
    %\end{psmallmatrix}\*d\vec{s}.\]
    Man kann komplexe Parameterintegrale so schreiben: 
    \[\int_{\gamma}f(z)dz=\int_{b}^a f(\gamma(t))\gamma'(t) dt\]
    %Sei $\vfunc{\text{ReIm}}{\bC}{\bR^2}{x+\imag y}{\begin{pmatrix}x\\y\end{pmatrix}}$, und\\ $$\vfunc{\operatorname{atan2}}{]-\pi,\pi]^2\setminus \qty{(0,0)}}{\bR }{(x,y)}{\begin{cases}
 %\arctan\left(\frac{y}{x}\right) &\text{if } x > 0, \\
 %\frac{\pi}{2} - \arctan\left(\frac{x}{y}\right) &\text{für } y > 0, \\
 %-\frac{\pi}{2} - \arctan\left(\frac{x}{y}\right) &\text{für } y < 0, \\
 %\arctan\left(\frac{y}{x}\right) \pm \pi &\text{für } x < 0.
%\end{cases}}$$
\begin{itemize}
\item[a)]
    für das erste Integral beobachen wir, dass im Einheitskreis wo $\gamma_1$ liegt, gilt:
    \[\overline{z}=\frac{1}{z}\]
     daraus folgt:
     \[\int_{\gamma_1} \frac{\log(\imag \log \overline z)}{z} dz 
     =\int_{\gamma_1} \frac{\log(\imag \log \frac{1}{z})}{z} dz \]
     Wir verwenden die parametrizierung und ihre Ableitung $\gamma_1'(t)=\imag e^{\imag t}$
      \[\int_{\frac{\pi}{2}}^\pi \frac{\log(\imag \log \frac{1}{e^{\imag t}})}{e^{\imag t}} \imag e^{\imag t} dt 
      =\int_{\frac{\pi}{2}}^\pi \log(\imag \log \frac{1}{e^{\imag t}}) \imag dt 
      =\int_{\frac{\pi}{2}}^\pi \log(\imag \log e^{-\imag t}) \imag dt 
      =\int_{\frac{\pi}{2}}^\pi \log(\imag( -\imag t)) \imag dt 
      \]
      \[=\int_{\frac{\pi}{2}}^\pi \log(t) \imag dt
      = \imag\int_{\frac{\pi}{2}}^\pi \log(t) dt =\imag\left[t \log(t)-t\right]_{t=\frac{\pi}{2}}^{t=\pi}
      =\imag \left(\pi \log(\pi)-\pi-\left(\frac{\pi}{2} \log(\frac{\pi}{2})-\frac{\pi}{2}\right)\right)\]
      \[=\imag \left(\pi \log(\pi)-\frac{\pi}{2} \log(\frac{\pi}{2}) -\frac{\pi}{2}\right)
      =-\imag \frac{1}{2} \pi \left(1+\log \left(\frac{\pi }{2}\right)-2 \log (\pi )\right)
      =-\imag \frac{1}{2} \pi  \left(1+\log \left(\frac{1}{2 \pi }\right)\right)
      =\imag \frac{1}{2} \pi \left(\log(2\pi)-1\right)
      \]
 \item[b)]
 \[\gamma_2'(t)=\imag\]

 Also \[\int_{\gamma_2}\frac{1}{z^2}dz
 =\int_{-1}^{2}\frac{1}{\left(1+\imag t\right)^2}\imag dt
 =\imag \int_{-1}^{2}\frac{1}{1-t^2+2\imag t}dt
 =\imag \int_{-1}^{2}\frac{1-t^2-2\imag t}{\pr{1-t^2}^2+\pr{2 t}^2}dt
 \]
 \[ =\imag \int_{-1}^{2}\frac{1-t^2-2\imag t}{\pr{t^4-2 t^2+1}+4t^2}dt
 =\imag \int_{-1}^{2}\frac{1-t^2-2\imag t}{t^4+2 t^2+1}dt
 =\imag \int_{-1}^{2}\frac{1-t^2-2\imag t}{\pr{t^2+1}^2}dt
 %=\imag \int_{-1}^{2}\frac{1-t^2-2\imag t}{\abs{t^2+1}}dt
 \]
 %Wir substituieren $t^2+1=u\>dt 2t=du$
 \[=\imag \int_{-1}^{2}\frac{1-t^2-2\imag t}{\pr{t^2+1}^2}dt
 =\pr{ \imag\int_{-1}^{2}\frac{1-t^2}{\pr{t^2+1}^2}dt+\int_{-1}^{2}\frac{2t}{\pr{t^2+1}^2}dt}
 = \imag\int_{-1}^{2}\frac{1-t^2}{\pr{t^2+1}^2}dt+\int_{-1}^{2}\frac{2t}{\pr{t^2+1}^2}dt
 \]
 \[= -\pr{\imag\int_{-1}^{2}\frac{1}{\pr{t^2+1}}-\frac{2}{\pr{t^2+1}^2}dt}+\int_{-1}^{2}\frac{2t}{\pr{t^2+1}^2}dt
 \]
 \[= -\imag\pr{\int_{-1}^{2}\frac{1}{\pr{t^2+1}}dt-\int_{-1}^{2}\frac{2}{\pr{t^2+1}^2}dt}+\int_{-1}^{2}\frac{2t}{\pr{t^2+1}^2}dt\]
 \[= -\imag\pr{\int_{-1}^{2}\frac{1}{\pr{t^2+1}}dt-2\pr{\br{\frac{t}{2\pr{t^2+1}}}_{t=-1}^{t=2}+\frac{1}{2}\int_{-1}^{2}\frac{1}{t^2+1}dt}}+\int_{-1}^{2}\frac{2t}{\pr{t^2+1}^2}dt\]
  \[= \imag\pr{2\pr{\br{\frac{t}{2\pr{t^2+1}}}_{t=-1}^{t=2}}}+\int_{-1}^{2}\frac{2t}{\pr{t^2+1}^2}dt\]
  \[= \imag\pr{\br{\frac{t}{\pr{t^2+1}}_{t=-1}^{t=2}}}+\int_{-1}^{2}\frac{2t}{\pr{t^2+1}^2}dt\]
  \[= \imag\pr{\frac{2}{\pr{2^2+1}}-\frac{-1}{\pr{(-1)^2+1}}}+\int_{-1}^{2}\frac{2t}{\pr{t^2+1}^2}dt\]
  \[= \imag\pr{\frac{2}{5}+\frac{1}{2}}+\int_{-1}^{2}\frac{2t}{\pr{t^2+1}^2}dt\]
  \[= \imag\frac{9}{10}+\int_{-1}^{2}\frac{2t}{\pr{t^2+1}^2}dt\]
  \[= \imag\frac{9}{10}+\int_{-1}^{2}\frac{\partial_t \pr{t^2+1}}{\pr{t^2+1}^2}dt\]
  \[= \imag\frac{9}{10}+\br{-\frac{1}{\pr{t^2+1}}}_{t=-1}^{t=2}
  =\imag\frac{9}{10}-\frac{1}{5}+\frac{1}{2}
  =\imag\frac{9}{10}+\frac{3}{10}\]
% \[=\imag\int_{-1}^{2}\pr{\frac{2}{t^2+1}-1}dt+\int_{-1}^{2}\frac{2t}{t^2+1}dt\]


\end{itemize}
    %\[\frac{\log(\imag \log \overline z)}{z} \]
    %$z\>x+\imag y$
    %\[\frac{\log(\imag \log \overline {(x+\imag y)})}{x+\imag y} =\frac{\log(\imag \log (x-\imag y))}{x+\imag y}\]
    %Wir können die Logarithm in imaginär und reellenteilen trennen \(\log(x+\imag y)=\log(\sqrtnorm{x,y}+\imag \operatorname{atan2}(x,y))\)
    %\[\frac{\log(\imag \left(\log\left(\sqrtnorm{x,y}\right)+\imag \operatorname{atan2}(x,y)\right))}{x+\imag y}
    %=\frac{\log(\imag \log\left(\sqrtnorm{x,y}\right)-\operatorname{atan2}(x,y))}{x+\imag y}
    %\]
    
    
\end{subsection}
\end{section}
\begin{section}{Aufgabe}%2
Werten Sie das Integral
\[\int_{\gamma}\frac{z}{1-z^2}dz,\quad \gamma:t\mapsto e^{\imag t},\frac{\pi}{4}<t<\frac{3 \pi}{4}\]
mit geeigneten Stammfunktionen aus.
    \begin{subsection}{Antwort}
        Wir können $z^2=u$ substituieren mit $\frac{du}{dz}=2z$.
        \[\int \frac{1}{2\pr{1-u}}du=-\frac{1}{2}\log(1-u)=-\frac{1}{2}\log(1-z^2).\]
        Wir haben eine Stammfunktion, die ihre Definitionsbereich $1-z^2\notin \mathbb G=\qty{x+\imag y\in \bC \mid x>0 \lor y\neq 0}$ gelten soll.\\
        für $z^2$, das Einheitskreissegment zwischen $\frac{\pi}{4}$ und $\frac{3\pi}{4}$ wird auf das Einheitskreissegment zwischen $\frac{\pi}{2}$ und $\frac{3 \pi}{2}$ abgebildet.
        $-1$ ist teil dieses Bereich, das ist aber eine Polstelle das Integral.
        Die Integral ist nicht definiert und hat keine Stammfunktion für diese Definitionsbereich.
        %$\frac{z}{1-z^2}$ ist eine Möbiustransformation\\
        %3 Punkten der Einheitskreis:
        %$\frac{1}{1-1}"="\infty$
        %$\frac{-1}{1-1}"="\infty$
    \end{subsection}
\end{section}
\begin{section}{Aufgabe}
    Welche der folgenden Punktmengen sind Gebiete, welche sogar einfach zusammenhängend?
Entscheiden Sie mit kurzer Begrundung.
\begin{itemize}
    \item[a)] \(M_7=\qty{z\in\bC \mid \abs{\Re z}<1},\)
    \item[b)] \(M_8=\qty{\left(x,y\right) \in\bR^2 \mid x^2+y^2>1},\)
    \item[c)] \(M_9=\qty{\left(x,y,z\right) \in\bR^3 \mid x+y+z\neq 1},\)
    \item[d)] \(M_{10}=\bR^3 \setminus \qty{\left(x,y,z\right) \in\bR^3 \mid y=z= 1},\)
    \item[e)] \(M_{11}=\bC \setminus \qty{z \in\bC \mid \abs{\Re z}\leq 1 \text{und} \Im z=0}.\)
\end{itemize}
      \begin{subsection}{Antwort}
        \begin{itemize}
\item[a)]
Nicht zussamenhängend da $x>1$ und $-x<-1$ getrennt ist.
\item[b)] 
            \end{itemize}
    \end{subsection}  
\end{section}
\end{document}
