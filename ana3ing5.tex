\documentclass{scrartcl}
\def\wochenumber{5}
\usepackage{amsmath}
\usepackage{amsfonts}
\usepackage{tabto}
\usepackage{enumerate}
\usepackage[ngerman]{babel}
%\usepackage[T1]{fontenc}
\usepackage{lmodern}
\usepackage[utf8]{inputenc}
\usepackage{tabularx}
\usepackage[paper=a4paper,left=25mm,right=25mm,top=20mm,bottom=20mm]{geometry}
\usepackage{nccmath}
\usepackage{xcolor}
\usepackage{ragged2e}
\usepackage{graphics}
\usepackage{graphicx}
\usepackage{fancyhdr}
\usepackage{mathtools}  
\usepackage{float}
\usepackage{physics}
\usepackage{subcaption}
%\usepackage{wasysym}
%\newcommand*{\vpointer}{\vcenter{\hbox{\scalebox{2}{\Huge\pointer}}}}
\usepackage[linkcolor=blu,colorlinks=false]{hyperref}

%\usepackage{cfr-lm} %ich erinnere mich nicht wieso war das wichtig
\usepackage{newtxmath}
\usepackage{cleveref}
\renewcommand{\headrulewidth}{0.5pt}
\renewcommand{\footrulewidth}{0.5pt}
\usepackage{microtype}  % Minature font tweaks
\usepackage{cancel}
%\usepackage{cmathbb}
\usepackage{tikz,calc}
\usepackage{tkz-fct}  
\usepackage{tkz-euclide}
\usepackage{derivative}
\usepackage{ifthen}
\usetikzlibrary{decorations.pathreplacing,decorations.markings}
%\usetikzlibrary{decorations.markings}
\tikzset{
    set arrow inside/.code={\pgfqkeys{/tikz/arrow inside}{#1}},
    set arrow inside={end/.initial=>, opt/.initial=},
    /pgf/decoration/Mark/.style={
        mark/.expanded=at position #1 with
        {
            \noexpand\arrow[\pgfkeysvalueof{/tikz/arrow inside/opt}]{\pgfkeysvalueof{/tikz/arrow inside/end}}
        }
    },
    arrow inside/.style 2 args={
        set arrow inside={#1},
        postaction={
            decorate,decoration={
                markings,Mark/.list={#2}
            }
        }
    },
}
\tikzset{
  % style to apply some styles to each segment of a path
  on each segment/.style={
    decorate,
    decoration={
      show path construction,
      moveto code={},
      lineto code={
        \path [#1]
        (\tikzinputsegmentfirst) -- (\tikzinputsegmentlast);
      },
      curveto code={
        \path [#1] (\tikzinputsegmentfirst)
        .. controls
        (\tikzinputsegmentsupporta) and (\tikzinputsegmentsupportb)
        ..
        (\tikzinputsegmentlast);
      },
      closepath code={
        \path [#1]
        (\tikzinputsegmentfirst) -- (\tikzinputsegmentlast);
      },
    },
  },
  % style to add an arrow in the middle of a path
  mid arrow/.style={postaction={decorate,decoration={
        markings,
        mark=at position .5 with {\arrow[#1]{stealth}}
      }}},
}

\usepackage[german=quotes]{csquotes}
%\DeclareSymbolFont{eulerletters}{U}{zeur}{b}{n}
%\DeclareSymbolFont{cmrot1italics}{OT1}{cmr}{m}{it}
%\DeclareMathSymbol{\imag}{\mathord}{eulerletters}{`i}
%\DeclareMathSymbol{\imag}{\mathord}{cmrot1italics}{`i}
\DeclareSymbolFont{uiletters}{OT1}{cmr}{m}{ui}
\DeclareMathSymbol{\imag}{\mathalpha}{uiletters}{`i}
% Common shortcuts
\def\mbb#1{\mathbb{#1}}
\def\mfk#1{\mathfrak{#1}}
%\def\imag{\mathsf{i}}
\def\bN{\mbb{N}}
\def\bC{\mbb{C}}
\def\bR{\mbb{R}}
\def\bQ{\mbb{Q}}
\def\bZ{\mbb{Z}}
\newcommand{\func}[3]{#1\colon#2\to#3}
\newcommand{\vfunc}[5]{\func{#1}{#2}{#3},\quad#4\longmapsto#5}
\newcommand{\floor}[1]{\left\lfloor#1\right\rfloor}
\newcommand{\ceil}[1]{\left\lceil#1\right\rceil}
%Abstand zwischen Absätzen, Zeilenabstände
\voffset26pt 
\parskip6pt
%\parindent1cm  %Rückt erste Zeile eines neuen Absatzes ein
\usepackage{setspace}
\onehalfspacing
\title{Analysis III für Ingenieurwissenschaften}
\author{Juan Pardo Martin (397882)\and Tuan Kiet Nguyen (404029)\and Leonardo Nerini (414193)}
\let\*\cdot
\let\-\rightarrow
\let\_\Rightarrow
\let\>\Leftrightarrow
\definecolor{l}{rgb}{0.0, 0.5, 0.0}
\newcommand\ReIm[0]{\text{ReIm}}
\makeatletter
\newcommand\sqrtnorm[1]{
    \count@=0
    \sqrt{
    \@for\levar:=#1\do{
        \advance\count@ 1
        \ifnum\count@=1
        \else +
        \fi
        {\levar}^2
        }
    }
}
\makeatother
\let\c\textcolor
%\mathbb{R} 
%\begin{align*}
\titlehead
{
\begin{tabular}{ll}
\begin{minipage}{0.5\textwidth}
%\begin{figure}[H]
% \raggedright
 %\includegraphics[scale=0.04]{tu-logo}\\
%\end{figure}
\end{minipage}
\begin{minipage}{0.5\textwidth}
\begin{figure}[H]
\raggedleft
%\includegraphics[scale=0.04]{tu-logo}\\
\end{figure}
\end{minipage}
\end{tabular}\\
\\
  \small
      {
    Ana III Hausaufgabe, \wochenumber{} Woche\\
    Tutor:  David Sering  \\
	SS 2021
    }


}
%\usepackage{mathtools} 
\begin{document}
\maketitle
%\author
\begin{section}{Aufgabe}%Aufgabe 1
    (Hausaufgabe zum Stoff der Ubung 4)\\
    Gegeben seien folgende zwei Kurven:
    \begin{align*}
        \gamma_1&: t\mapsto e^{\imag t},\quad \frac{\pi}{2}\leq t \leq \pi,\\
        \gamma_2&: t\mapsto 1+\imag t,\quad -1\leq t \leq 2.
    \end{align*}
Berechnen Sie die folgenden komplexen Kurvenintegrale mit der Definition als Parameterintegrale:
\[\text{a)} \int_{\gamma_1}\frac{\log(\imag \log \overline z)}{z}dz,\text{ b) }\int_{\gamma_2}\frac{1}{z^2}dz\]

\begin{subsection}{Antwort}
    %der Zusammenhang zwischen Kurvenintegral und Vektorfeldern: 
    %\[\int_C f(z)dz=\int_C \begin{psmallmatrix}
    %    u\\-v
    %\end{psmallmatrix}\*d\vec{s}+\imag \int_C \begin{psmallmatrix}
    %    v\\u
    %\end{psmallmatrix}\*d\vec{s}.\]
    Man kann komplexe Parameterintegrale so schreiben: 
    \[\int_{\gamma}f(z)dz=\int_{b}^a f(\gamma(t))\gamma'(t) dt\]
    %Sei $\vfunc{\text{ReIm}}{\bC}{\bR^2}{x+\imag y}{\begin{pmatrix}x\\y\end{pmatrix}}$, und\\ $$\vfunc{\operatorname{atan2}}{]-\pi,\pi]^2\setminus \qty{(0,0)}}{\bR }{(x,y)}{\begin{cases}
 %\arctan\left(\frac{y}{x}\right) &\text{if } x > 0, \\
 %\frac{\pi}{2} - \arctan\left(\frac{x}{y}\right) &\text{für } y > 0, \\
 %-\frac{\pi}{2} - \arctan\left(\frac{x}{y}\right) &\text{für } y < 0, \\
 %\arctan\left(\frac{y}{x}\right) \pm \pi &\text{für } x < 0.
%\end{cases}}$$
\begin{itemize}
\item[a)]
    für das erste Integral beobachen wir, dass im Einheitskreis wo $\gamma_1$ liegt, gilt:
    \[\overline{z}=\frac{1}{z}\]
     daraus folgt:
     \[\int_{\gamma_1} \frac{\log(\imag \log \overline z)}{z} dz 
     =\int_{\gamma_1} \frac{\log(\imag \log \frac{1}{z})}{z} dz \]
     Wir verwenden die parametrizierung und ihre Ableitung $\gamma_1'(t)=\imag e^{\imag t}$
      \[\int_{\frac{\pi}{2}}^\pi \frac{\log(\imag \log \frac{1}{e^{\imag t}})}{e^{\imag t}} \imag e^{\imag t} dt 
      =\int_{\frac{\pi}{2}}^\pi \log(\imag \log \frac{1}{e^{\imag t}}) \imag dt 
      =\int_{\frac{\pi}{2}}^\pi \log(\imag \log e^{-\imag t}) \imag dt 
      =\int_{\frac{\pi}{2}}^\pi \log(\imag( -\imag t)) \imag dt 
      \]
      \[=\int_{\frac{\pi}{2}}^\pi \log(t) \imag dt
      = \imag\int_{\frac{\pi}{2}}^\pi \log(t) dt =\imag\left[t \log(t)-t\right]_{t=\frac{\pi}{2}}^{t=\pi}
      =\imag \left(\pi \log(\pi)-\pi-\left(\frac{\pi}{2} \log(\frac{\pi}{2})-\frac{\pi}{2}\right)\right)\]
      \[=\imag \left(\pi \log(\pi)-\frac{\pi}{2} \log(\frac{\pi}{2}) -\frac{\pi}{2}\right)
      =-\imag \frac{1}{2} \pi \left(1+\log \left(\frac{\pi }{2}\right)-2 \log (\pi )\right)
      =-\imag \frac{1}{2} \pi  \left(1+\log \left(\frac{1}{2 \pi }\right)\right)
      =\imag \frac{1}{2} \pi \left(\log(2\pi)-1\right)
      \]
 \item[b)]
 \[\gamma_2'(t)=\imag\]
 Also \[\int_{\gamma_2}\frac{1}{z^2}dz
 =\int_{-1}^{2}\frac{1}{\left(1+\imag t\right)^2}\imag dt
 =\imag \int_{-1}^{2}\frac{1}{1-t^2+2\imag t}dt
 =\imag \int_{-1}^{2}\frac{1-t^2-2\imag t}{\qty(1-t^2)^2+\qty(2 t)^2}dt
 \]
 \[ =\imag \int_{-1}^{2}\frac{1-t^2-2\imag t}{\qty(t^4-2 t^2+1)+4t^2}dt
 =\imag \int_{-1}^{2}\frac{1-t^2-2\imag t}{t^4+2 t^2+1}dt
 =\imag \int_{-1}^{2}\frac{1-t^2-2\imag t}{\qty(t^2+1)^2}dt
 %=\imag \int_{-1}^{2}\frac{1-t^2-2\imag t}{\abs{t^2+1}}dt
 \]
 %Wir substituieren $t^2+1=u\>dt 2t=du$
 \[=\imag \int_{-1}^{2}\frac{1-t^2-2\imag t}{\qty(t^2+1)^2}dt
 =\qty( \imag\int_{-1}^{2}\frac{1-t^2}{\qty(t^2+1)^2}dt+\int_{-1}^{2}\frac{2t}{\qty(t^2+1)^2}dt)
 = \imag\int_{-1}^{2}\frac{1-t^2}{\qty(t^2+1)^2}dt+\int_{-1}^{2}\frac{2t}{\qty(t^2+1)^2}dt
 \]
 \[= -\qty(\imag\int_{-1}^{2}\frac{1}{\qty(t^2+1)}-\frac{2}{\qty(t^2+1)^2}dt)+\int_{-1}^{2}\frac{2t}{\qty(t^2+1)^2}dt
 \]
 \[= -\imag\qty(\int_{-1}^{2}\frac{1}{\qty(t^2+1)}dt-\int_{-1}^{2}\frac{2}{\qty(t^2+1)^2}dt)+\int_{-1}^{2}\frac{2t}{\qty(t^2+1)^2}dt\]
 \[= -\imag\qty(\int_{-1}^{2}\frac{1}{\qty(t^2+1)}dt
 -2\qty(\qty[\frac{t}{2\qty(t^2+1)}_{t=-1}^{t=2}+\frac{1}{2}\int_{-1}^{2}\frac{1}{t^2+1}dt]))
 +\int_{-1}^{2}\frac{2t}{\qty(t^2+1)^2}
 dt\]
  \[= \imag\qty(2\qty(\qty[\frac{t}{2\qty(t^2+1)}]_{t=-1}^{t=2}))+\int_{-1}^{2}\frac{2t}{\qty(t^2+1)^2}dt\]
  \[= \imag\qty(\qty[\frac{t}{\qty(t^2+1)}_{t=-1}^{t=2}])+\int_{-1}^{2}\frac{2t}{\qty(t^2+1)^2}dt\]
  \[= \imag\qty(\frac{2}{\qty(2^2+1)}-\frac{-1}{\qty((-1)^2+1)})+\int_{-1}^{2}\frac{2t}{\qty(t^2+1)^2}dt\]
  \[= \imag\qty(\frac{2}{5}+\frac{1}{2})+\int_{-1}^{2}\frac{2t}{\qty(t^2+1)^2}dt\]
  \[= \imag\frac{9}{10}+\int_{-1}^{2}\frac{2t}{\qty(t^2+1)^2}dt\]
  \[= \imag\frac{9}{10}+\int_{-1}^{2}\frac{\partial_t \qty(t^2+1)}{\qty(t^2+1)^2}dt\]
 \[= \imag\frac{9}{10}+\qty[-\frac{1}{\qty(t^2+1)}]_{t=-1}^{t=2}
  =\imag\frac{9}{10}-\frac{1}{5}+\frac{1}{2}
  =\imag\frac{9}{10}+\frac{3}{10}\]

\end{itemize}
    %\[\frac{\log(\imag \log \overline z)}{z} \]
    %$z\>x+\imag y$
    %\[\frac{\log(\imag \log \overline {(x+\imag y)})}{x+\imag y} =\frac{\log(\imag \log (x-\imag y))}{x+\imag y}\]
    %Wir können die Logarithm in imaginär und reellenteilen trennen \(\log(x+\imag y)=\log(\sqrtnorm{x,y}+\imag \operatorname{atan2}(x,y))\)
    %\[\frac{\log(\imag \left(\log\left(\sqrtnorm{x,y}\right)+\imag \operatorname{atan2}(x,y)\right))}{x+\imag y}
    %=\frac{\log(\imag \log\left(\sqrtnorm{x,y}\right)-\operatorname{atan2}(x,y))}{x+\imag y}
    %\]
\end{subsection}
\end{section}
\begin{section}{Aufgabe}%2
Werten Sie das Integral
\[\int_{\gamma}\frac{z}{1-z^2}dz,\quad \gamma:t\mapsto e^{\imag t},\frac{\pi}{4}<t<\frac{3 \pi}{4}\]
mit geeigneten Stammfunktionen aus.
    \begin{subsection}{Antwort}
        Wir können $z^2=u$ substituieren mit $\frac{du}{dz}=2z$.
        \[\int \frac{1}{2\qty(1-u)}du=-\frac{1}{2}\log(1-u)=-\frac{1}{2}\log(1-z^2).\]
        Wir haben eine Stammfunktion, die ihre Definitionsbereich $1-z^2\notin \mathbb G=\qty{x+\imag y\in \bC \mid x>0 \lor y\neq 0}$ gelten soll.\\
        Jedoch ist $-\frac{1}{2}\log(a(1-u)),a\in\bR\setminus\qty{0}$ auch eine Stammfunktion von $\frac{1}{2\qty(1-u)}$, die verschiedenen Definitionenbereichen haben kann.
        Wir setzen $a=1$, und können sehen, dass wenn $1-u$ reell und $<0$ ist, ist es nicht definiert.
        Das ist ok, weil unsere kurve berüht die Positive reell Achse nicht. Also wir können die Stammfunktion anwenden.
        \[-\frac{1}{2}\log(1-e^{\imag \frac{3 \pi}{2}})+\frac{1}{2}\log(e^{1-\imag \frac{\pi}{2}})
        =\qty(-\frac{\log (2)}{4}-\frac{i \pi }{8})-\qty(-\frac{\log (2)}{4}+\frac{i \pi }{8})\]
        \[- \imag \frac{\pi}4\]

        %für $z^2$, das Einheitskreissegment zwischen $\frac{\pi}{4}$ und $\frac{3\pi}{4}$ wird auf das Einheitskreissegment zwischen $\frac{\pi}{2}$ und $\frac{3 \pi}{2}$ abgebildet.
        %$-1$ ist teil dieses Bereich, das ist aber eine Polstelle das Integral.
        %Die Integral ist nicht definiert und hat keine Stammfunktion für diese Definitionsbereich.
        %$\frac{z}{1-z^2}$ ist eine Möbiustransformation\\
        %3 Punkten der Einheitskreis:
        %$\frac{1}{1-1}"="\infty$
        %$\frac{-1}{1-1}"="\infty$
    \end{subsection}
\end{section}
\begin{section}{Aufgabe}
    Welche der folgenden Punktmengen sind Gebiete, welche sogar einfach zusammenhängend?
Entscheiden Sie mit kurzer Begrundung.
\begin{itemize}
    \item[a)] \(M_7=\qty{z\in\bC \mid \abs{\Re z}<1},\)
    \item[b)] \(M_8=\qty{\left(x,y\right) \in\bR^2 \mid x^2+y^2>1},\)
    \item[c)] \(M_9=\qty{\left(x,y,z\right) \in\bR^3 \mid x+y+z\neq 1},\)
    \item[d)] \(M_{10}=\bR^3 \setminus \qty{\left(x,y,z\right) \in\bR^3 \mid y=z= 1},\)
    \item[e)] \(M_{11}=\bC \setminus \qty{z \in\bC \mid \abs{\Re z}\leq 1 \text{ und } \Im z=0}.\)
\end{itemize}
      \begin{subsection}{Antwort}
        \begin{itemize}
\item[a)]
Diese Punktmenge soll eine Ebene zwischen den zwei Geraden, die parallel zur Imaginärchse sind und jeweils durch den Punkt $(1,0)$ und $(-1,0)$ auf der reellen Achse durchgehen, darstellen (Punkte auf dieser zwei Geraden gehören nicht zur der Punktmenge).\\
Diese Menge ist offen, vorstellungsweise einfach zusammenhängend \\
$\rightarrow$ ist ein Gebiet, das sogar einfach zusammenhängend ist. \\

\item[b)] Diese Punktmenge soll die komplexe Menge ohne den Einheitskreis und auch ohne alle Punkte, die er umschließt, sein. \\
Diese Menge ist dann offen und vorstellungsweise auch zusammenhängend. Daher ist sie ein Gebiet. Sie ist aber nicht einfach zusammenhängend, weil z.B: wir nehmen den Kreis mit dem Radius 2 und der Mittelpunkt liegt im Ursprung an. Der Kreis gehört zur unseren Punktmenge, aber der zieht sich zusammen um den Ursprung, der aber nicht zur unseren Menge gehört.\\
\item[c)] Es ist auch nicht zussamenhängend da es eine Ebene gibt die der Raum in 2 Teilen trennt. z.B man kann keine Pfad finden für $(5,0,0)$ und $(-5,0,0)$.
\item[d)] Es ist zussamenhängend, jedoch nicht einfach zussamenhängend da es eine lücke gibt, die Linie $y=z=1$.
Man kann eine geschloßene Bahn konstruiren z.B $\left(
\begin{array}{c}
 x\\\\1+\cos(\phi)\\\\1+\sin(\phi)
\end{array}
\right),\phi\in[0,2\pi[$ die sich nicht auf einen Punkt zussamenziehen lässt.
Da es zussamenhängend ist und Offen, da der Rand von die Menge nicht erhalten ist, ist es ein Gebiet.
\item[e)] Diese Menge ist zussamenhängend jedoch auch nicht einfach zussamenhängend, da Kreise in der Menge auf der Ursprung sich nicht zussamenziehen lassen. 
Da es zussamenhängend ist und Offen, da der Rand von die Menge auch nicht erhalten ist, ist es ein Gebiet.
%https://cdn.discordapp.com/attachments/766357188813783080/843549568669843497/Screenshot_20210516-200026_Discord.jpg
%https://cdn.discordapp.com/attachments/766357188813783080/843549569132527636/Screenshot_20210516-200213_Discord.jpg
%https://math.codidact.com/posts/281864
%https://proofwiki.org/wiki/Definition:Simply_Connected
            \end{itemize}
    \end{subsection}  
\end{section}
\end{document}
