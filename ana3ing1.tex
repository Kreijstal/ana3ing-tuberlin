\documentclass{scrartcl}
\usepackage{amsmath}
\usepackage{amsfonts}
\usepackage{tabto}
\usepackage{enumerate}
\usepackage[ngerman]{babel}
\usepackage[T1]{fontenc}
\usepackage{lmodern}
\usepackage[utf8]{inputenc}
\usepackage{tabularx}
\usepackage[paper=a4paper,left=25mm,right=25mm,top=20mm,bottom=20mm]{geometry}
\usepackage{nccmath}
\usepackage{xcolor}
\usepackage{ragged2e}
\usepackage{graphics}
\usepackage{graphicx}
\usepackage{fancyhdr}
\usepackage{mathtools}  
\usepackage{float}
\usepackage{physics}
\usepackage[linkcolor=blu,colorlinks=false]{hyperref}
\usepackage{cleveref}
\renewcommand{\headrulewidth}{0.5pt}
\renewcommand{\footrulewidth}{0.5pt}
\usepackage{microtype}  % Minature font tweaks

% Common shortcuts
\def\mbb#1{\mathbb{#1}}
\def\mfk#1{\mathfrak{#1}}

\def\bN{\mbb{N}}
\def\bC{\mbb{C}}
\def\bR{\mbb{R}}
\def\bQ{\mbb{Q}}
\def\bZ{\mbb{Z}}
\newcommand{\func}[3]{#1\colon#2\to#3}
\newcommand{\vfunc}[5]{\func{#1}{#2}{#3},\quad#4\longmapsto#5}
\newcommand{\floor}[1]{\left\lfloor#1\right\rfloor}
\newcommand{\ceil}[1]{\left\lceil#1\right\rceil}
%Abstand zwischen Absätzen, Zeilenabstände
\voffset26pt 
\parskip6pt
%\parindent1cm  %Rückt erste Zeile eines neuen Absatzes ein
\usepackage{setspace}
\onehalfspacing
\title{Analysis III für Ingenieurwissenschaften}
\author{Juan Pardo Martin (397882)\and Tuan Kiet Nguyen (404029)\and Leonardo Nerini (414193)}
\let\*\cdot
\let\-\rightarrow
\let\_\Rightarrow
\let\>\Leftrightarrow
\definecolor{l}{rgb}{0.0, 0.5, 0.0}
\let\c\textcolor
%\mathbb{R} 
%\begin{align*}
\titlehead
{
\begin{tabular}{ll}
\begin{minipage}{0.5\textwidth}
%\begin{figure}[H]
% \raggedright
 %\includegraphics[scale=0.04]{tu-logo}\\
%\end{figure}
\end{minipage}
\begin{minipage}{0.5\textwidth}
\begin{figure}[H]
\raggedleft
%\includegraphics[scale=0.04]{tu-logo}\\
\end{figure}
\end{minipage}
\end{tabular}\\
\\
  \small
      {
    Ana III Hausaufgabe, 1 Woche\\
    Tutor:  David Sering  \\
	SS 2021
    }

}
\begin{document}

\maketitle
%\author

\begin{section}{Aufgabe}%Aufgabe 1
Stellen Sie die folgenden Terme komplexer Funktionen in der Form \(u(x, y)+iv(x, y)\) dar,
wobei \(z = x + iy\) mit \(x, y \in \mathbb R\) eine komplexe Zahl und \(u, v : \mathbb R^2 \rightarrow \mathbb R\) reelle Funktionen
sind. Geben Sie hierbei die Funktionsterme \(u(x, y)\) und \(v(x, y)\) explizit an.
\begin{center}
a) \(e^{-i z^2}\),    b) \(z\overline{z}^2+\overline{z}z^2\)
\end{center}
\begin{itemize}
\item[a)]
Wir ersetzen \(z=x+iy\), wobei \(x,y\in \mathbb R\)
\[e^{-i (x+i y)^2}\]
Wir multiplizieren die Exponente aus.
\[e^{-i x^2+2 x y+i y^2}\]
Wir klamern wir die imaginäre teil aus.
\[e^{2 x y} e^{-i x^2+i y^2}\]
Wir anwenden die Eulersche Identität
\[e^{2 x y} \left(\cos \left(x^2-y^2\right)-i \sin \left(x^2-y^2\right)\right)\]
\[\left(\underbrace{e^{2 x y} \cos \left(x^2-y^2\right)}_{u(x,y)}-i \underbrace{e^{2 x y}\sin \left(x^2-y^2\right)}_{v(x,y)}\right)\]
\newpage
\item[b)]
\[z\overline{z}^2+\overline{z}z^2\]
\[(x+i y) (x-i y)^2+(x+i y)^2 (x-i y)\]
\[x^3-i x^2 y+x y^2-i y^3+x^3+i x^2 y+x y^2+i y^3\]
\[\underbrace{2 x^3 + 2 x y^2}_{u(x,y)}+\underbrace{0}_{v(x,y)}\]
\end{itemize}
\end{section}
\begin{section}{Aufgabe}%Aufgabe 2
Konstruieren Sie eine auf ganz $\bC$ definierte Funktion $f$, die genau auf der Geraden \(\qty{z \in
\mathbb C | \Re z = 2 \Im z}\) komplex differenzierbar ist 
und an der Stelle \(2 + i\) den Wert 2 annimmt. Der Funktionsterm 
$f(z)$ ist als Term in der Variablen $z$ zu schreiben.

\[f(2+i)=2\]
Wir nutzen die Folgende funktion als Ansatz:
%\[\vfunc{f}{\bC}{\bC}{z}{2}\]
%zu einfach? oder muss exlplizit von z abhängen?
\[\vfunc{f}{\bC}{\bC}{z}{\frac{\Re(z)^2}{2}+i(\Im(z)^2)}+i\]
\[\partial_x u = x = 2y = \partial_y v\]
\[\partial_y u = 0 = 0 = - \partial_x v\]
Die Cauchy Riemann DGL bedingungen werden nur erfüllt für punkten die an der Gerade $ \Re z = 2 \Im z$ liegen.
Da alle die partielle Ableitungen stetig sind, ist es im sinne von $\bR^2$ Reelle differenzierbar.
An der stelle $(2+i)$ unsere funktion liefert $\frac{4}{2}-i+i=2$, also stimmt es.
$f$ entspricht die Bedingungen.


\end{section}

\begin{section}{Aufgabe}%Aufgabe 3
Gegeben ist die komplexe Funktion g mit
\[h(z)=\begin{cases} 
\frac{\Re\left(z^2\right)}{z} & \text{für } z\neq 0 \\
 0 & \text{für } z=0 \\
   \end{cases}\]
Zeigen Sie, dass die Funktion g im Punkt 0 die Cauchy-Riemann-Differentialgleichungen
erfullt, aber nicht komplex-differenzierbar ist. 
\[h(x+iy)=\begin{cases} 
\frac{\Re\left((x+iy)^2\right)}{(x+iy)} & \text{für } z\neq 0 \\
 0 & \text{für } z=0 \\
   \end{cases}\]

\[h(x+iy)=\begin{cases} 
\frac{\Re\left(x^2 + 2 i x y - y^2\right)}{(x+iy)} & \text{für } z\neq 0 \\
 0 & \text{für } z=0 \\
   \end{cases}\]

\[h(x+iy)=\begin{cases} 
\frac{x^2 - y^2}{(x+iy)} & \text{für } z\neq 0 \\
 0 & \text{für } z=0 \\
   \end{cases} \]

\[h(x+iy)=\begin{cases} 
\frac{(x^2 - y^2)(x-iy)}{(x^2+y^2)} & \text{für } z\neq 0 \\
 0 & \text{für } z=0 \\
   \end{cases} \]

\[u(x,y)=\begin{cases} 
\frac{x(x^2 - y^2)}{(x^2+y^2)} & \text{für } 
     \begin{psmallmatrix}x \\y \end{psmallmatrix}
     \neq 
     \begin{psmallmatrix}0\\0 \end{psmallmatrix}
      \\
 0 & \text{für } 
     \begin{psmallmatrix}x \\y \end{psmallmatrix}
     =
     \begin{psmallmatrix}0\\0 \end{psmallmatrix}
     \\
   \end{cases}\]
   
\[v(x,y)=\begin{cases} 
-\frac{y(x^2 - y^2)}{(x^2+y^2)} & \text{für } 
     \begin{psmallmatrix}x \\y \end{psmallmatrix}
     \neq 
     \begin{psmallmatrix}0\\0 \end{psmallmatrix}
      \\
 0 & \text{für } 
     \begin{psmallmatrix}x \\y \end{psmallmatrix}
     =
     \begin{psmallmatrix}0\\0 \end{psmallmatrix}
     \\
   \end{cases}\]
Die Cauchy-Riemann bedingungen lauten.
\end{section}
\begin{align}
\partial_x u &\overset{!}{=} \partial_y v \label{e}\\ 
\partial_y u &\overset{!}{=} - \partial_x v \label{auf3bed2}
\end{align}
für \(\partial_x u\) im Punkt 0
\[\partial_x u(x,0)=\lim_{h\rightarrow 0}\frac{u(h,0)-\overbrace{u(0,0)}^{0}}{h}=\frac{h^3}{h^3}=1\]
\[\partial_y v(0,y)=\lim_{h\rightarrow 0}\frac{v(0,h)-\overbrace{v(0,0)}^{0}}{h}=\frac{h^3}{h^3}=1\]
Somit \eqref{e} gilt.
\[\partial_y u(0,y)=\lim_{h\rightarrow 0}\frac{\overbrace{u(0,h)}^0-\overbrace{u(0,0)}^{0}}{h}=0\]
\[\partial_x v(0,y)=\lim_{h\rightarrow 0}\frac{\overbrace{v(h,0)}^0-\overbrace{v(0,0)}^{0}}{h}=0\]
Daraus folgt \eqref{auf3bed2} stimmt auch
Die Cauchy-Riemannsche bedingungen stimmen, aber es ist nicht hinreichend für Komplexe differenzierbarkeit, wir müssen zeigen dass die Abbildung
\[f:\begin{pmatrix}x \\y \end{pmatrix}\mapsto \begin{pmatrix}u(x,y) \\v(x,y) \end{pmatrix}\]
Reelle differenzierbar sein.
Wir untersuchen die differenzierbarkeit an der stelle \(\begin{psmallmatrix}0\\0\end{psmallmatrix}\).
Wir berechnen die Funktionalmatrize an der stelle \(\begin{psmallmatrix}0\\0\end{psmallmatrix}\)
\[f'(0,0)=\begin{pmatrix}\partial_x u&\partial_y u\\\partial_x v&\partial_y v\end{pmatrix}=\begin{pmatrix}1&0\\0&1\end{pmatrix}\]
Wenn
\[\lim_{(\Delta x,\Delta y)\rightarrow(0,0)}\frac{\left|f(\Delta x,\Delta y)-f(0,0)-f'(0,0)\begin{psmallmatrix}\Delta x\\\Delta y\end{psmallmatrix}\right|}{\sqrt{(\Delta x)^2+(\Delta y)^2}}\]
0 wäre, dann wäre es komplexe differenzierbar.
\[\lim_{(\Delta x,\Delta y)\rightarrow(0,0)}\frac{\left|\begin{psmallmatrix}u(\Delta x,\Delta y)\\v(\Delta x,\Delta y)\end{psmallmatrix}-\overbrace{f(0,0)}^0-\begin{psmallmatrix}\Delta x\\\Delta y\end{psmallmatrix}\right|}{\sqrt{(\Delta x)^2+(\Delta y)^2}}=\lim_{(\Delta x,\Delta y)\rightarrow(0,0)}\frac{\left|\begin{psmallmatrix}\frac{(\Delta x)((\Delta x)^2 - (\Delta y)^2)}{((\Delta x)^2+(\Delta y)^2)}-\Delta x\\-\frac{(\Delta y)((\Delta x)^2 - (\Delta y)^2)}{((\Delta x)^2+(\Delta y)^2)}-\Delta y\end{psmallmatrix}\right|}{\sqrt{(\Delta x)^2+(\Delta y)^2}}\]
%\left(\frac{(\Delta x)((\Delta x)^2 - (\Delta y)^2)}{((\Delta x)^2+(\Delta y)^2)}-\Delta x\right)^2
%\left(-\frac{(\Delta y)((\Delta x)^2 - (\Delta y)^2)}{((\Delta x)^2+(\Delta y)^2)}-\Delta y\right)^2
%\[=\lim_{(\Delta x,\Delta y)\rightarrow(0,0)}\sqrt{\frac{1}{(\Delta x)^2+(\Delta y)^2}}}\]
\[=\lim_{(\Delta x,\Delta y)\rightarrow(0,0)}\sqrt{\frac{\left(\frac{(\Delta x)((\Delta x)^2 - (\Delta y)^2)}{((\Delta x)^2+(\Delta y)^2)}-\Delta x\right)^2+\left(-\frac{(\Delta y)((\Delta x)^2 - (\Delta y)^2)}{((\Delta x)^2+(\Delta y)^2)}-\Delta y\right)^2}{(\Delta x)^2+(\Delta y)^2}}\]
\[=\lim_{(\Delta x,\Delta y)\rightarrow(0,0)}\sqrt{\frac{\left(-\frac{2 (\Delta x) (\Delta y)^2}{(\Delta x)^2+(\Delta y)^2}\right)^2+\left(-\frac{2 (\Delta x)^2 (\Delta y)}{(\Delta x)^2+(\Delta y)^2}\right)^2}{(\Delta x)^2+(\Delta y)^2}}\]
\[=\lim_{(\Delta x,\Delta y)\rightarrow(0,0)}\sqrt{\frac{\left(\frac{2 (\Delta x) (\Delta y)^2}{(\Delta x)^2+(\Delta y)^2}\right)^2+\left(\frac{2 (\Delta x)^2 (\Delta y)}{(\Delta x)^2+(\Delta y)^2}\right)^2}{(\Delta x)^2+(\Delta y)^2}}\]
\[=\lim_{(\Delta x,\Delta y)\rightarrow(0,0)}\sqrt{\frac{\left(2 (\Delta x) (\Delta y)^2\right)^2+\left(2 (\Delta x)^2 (\Delta y)\right)^2}{\left((\Delta x)^2+(\Delta y)^2\right)^3}}\]
\[=\lim_{(\Delta x,\Delta y)\rightarrow(0,0)}2 \sqrt{\frac{\left((\Delta x) (\Delta y)^2\right)^2+\left((\Delta x)^2 (\Delta y)\right)^2}{\left((\Delta x)^2+(\Delta y)^2\right)^3}}\]
\[=\lim_{k\rightarrow \infty }2 \sqrt{\frac{\left((\frac{1}{k}) (\frac{1}{k})^2\right)^2+\left((\frac{1}{k})^2 (\frac{1}{k})\right)^2}{\left((\frac{1}{k})^2+(\frac{1}{k})^2\right)^3}}\]
\[=\lim_{k\rightarrow \infty }2 \sqrt{\frac{\left((\frac{1}{k})^3\right)^2+\left((\frac{1}{k})^3 \right)^2}{8(\frac{1}{k})^6}}\]
\[=\lim_{k\rightarrow \infty } \sqrt{\frac{(\frac{1}{k})^6}{(\frac{1}{k})^6}}=\sqrt{1}=1\neq0\]
Die Abbildung $f$ ist nicht Reelle differenzierbar an der Stelle \(\begin{psmallmatrix}0\\0\end{psmallmatrix}\)
Daraus folgt, dass $h$ nicht Komplexe differenzierbar ist.
Alternativ könnte man mit die Komplexen Differenzialquotient von $h$ an der stelle 0.
\[\lim_{\Delta z\rightarrow 0}\frac{h(0+\Delta z)+h(0)}{\Delta z}=\lim_{\Delta z\rightarrow 0}\frac{\Re((\Delta z)^2)}{(\Delta z)^2}\]
Wir nehmen 2 verschiedene 0 Folgen.
\(\Delta z=\lim_{k\rightarrow\infty}\frac{1}k\)
\[\lim_{k\rightarrow\infty}=\lim_{\Delta z\rightarrow 0}\frac{\Re((\frac 1 k)^2)}{(\frac 1 k)^2}=1\]
und \(\Delta z=\lim_{k\rightarrow\infty}\frac{1+i}k\)
\[\lim_{k\rightarrow\infty}=\lim_{\Delta z\rightarrow 0}\frac{\Re((\frac {1+i} k)^2)}{(\frac {1+i} k)^2}=0\]
Der Grenzwert ist nicht definiert und somit ist $h$ nicht komplex differenzierbar an der Stelle 0.

\end{document}
