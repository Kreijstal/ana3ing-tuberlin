\documentclass{scrartcl}
\usepackage{amsmath}
\usepackage{amsfonts}
\usepackage{tabto}
\usepackage{enumerate}
\usepackage[ngerman]{babel}
\usepackage[T1]{fontenc}
\usepackage{lmodern}
\usepackage[utf8]{inputenc}
\usepackage{tabularx}
\usepackage[paper=a4paper,left=25mm,right=25mm,top=20mm,bottom=20mm]{geometry}
\usepackage{nccmath}
\usepackage{xcolor}
\usepackage{ragged2e}
%\usepackage{graphics}
\usepackage{graphicx}
\usepackage{fancyhdr}
\usepackage{mathtools}  
\usepackage{float}
\usepackage{physics}
\usepackage[linkcolor=blu,colorlinks=false]{hyperref}

%\usepackage{breqn}
\usepackage{cfr-lm}
\usepackage{newtxmath}
\usepackage{cleveref}
\renewcommand{\headrulewidth}{0.5pt}
\renewcommand{\footrulewidth}{0.5pt}
\usepackage{microtype}  % Minature font tweaks
\usepackage{tkz-fct}  
\usepackage{tkz-euclide}
\usepackage{tikz,calc}
\usetikzlibrary{decorations.pathreplacing,decorations.markings}
%\usetikzlibrary{decorations.markings}
\tikzset{
    set arrow inside/.code={\pgfqkeys{/tikz/arrow inside}{#1}},
    set arrow inside={end/.initial=>, opt/.initial=},
    /pgf/decoration/Mark/.style={
        mark/.expanded=at position #1 with
        {
            \noexpand\arrow[\pgfkeysvalueof{/tikz/arrow inside/opt}]{\pgfkeysvalueof{/tikz/arrow inside/end}}
        }
    },
    arrow inside/.style 2 args={
        set arrow inside={#1},
        postaction={
            decorate,decoration={
                markings,Mark/.list={#2}
            }
        }
    },
}
%\usepackage{cmathbb}
%\DeclareSymbolFont{eulerletters}{U}{zeur}{b}{n}
%\DeclareSymbolFont{cmrot1italics}{OT1}{cmr}{m}{it}
%\DeclareMathSymbol{\imag}{\mathord}{eulerletters}{`i}
%\DeclareMathSymbol{\imag}{\mathord}{cmrot1italics}{`i}
\DeclareSymbolFont{uiletters}{OT1}{cmr}{m}{ui}
\DeclareMathSymbol{\imag}{\mathalpha}{uiletters}{`i}
% Common shortcuts
\def\mbb#1{\mathbb{#1}}
\def\mfk#1{\mathfrak{#1}}
%\def\imag{\mathsf{i}}
\def\bN{\mbb{N}}
\def\bC{\mbb{C}}
\def\bR{\mbb{R}}
\def\bQ{\mbb{Q}}
\def\bZ{\mbb{Z}}
\newcommand{\func}[3]{#1\colon#2\to#3}
\newcommand{\vfunc}[5]{\func{#1}{#2}{#3},\quad#4\longmapsto#5}
\newcommand{\floor}[1]{\left\lfloor#1\right\rfloor}
\newcommand{\ceil}[1]{\left\lceil#1\right\rceil}
%Abstand zwischen Absätzen, Zeilenabstände
\voffset26pt 
\parskip6pt
%\parindent1cm  %Rückt erste Zeile eines neuen Absatzes ein
\usepackage{setspace}
\onehalfspacing
\title{Analysis III für Ingenieurwissenschaften}
\author{Juan Pardo Martin (397882)\and Tuan Kiet Nguyen (404029)\and Leonardo Nerini (414193)}
\let\*\cdot
\let\-\rightarrow
\let\_\Rightarrow
\let\>\Leftrightarrow
\definecolor{l}{rgb}{0.0, 0.5, 0.0}
\let\c\textcolor
%\mathbb{R} 
%\begin{align*}
\titlehead
{
\begin{tabular}{ll}
\begin{minipage}{0.5\textwidth}
%\begin{figure}[H]
% \raggedright
 %\includegraphics[scale=0.04]{tu-logo}\\
%\end{figure}
\end{minipage}
\begin{minipage}{0.5\textwidth}
\begin{figure}[H]
\raggedleft
%\includegraphics[scale=0.04]{tu-logo}\\
\end{figure}
\end{minipage}
\end{tabular}\\
\\
  \small
      {
    Ana III Hausaufgabe, 3 Woche\\
    Tutor:  David Sering  \\
	SS 2021
    }


}
\begin{document}
\maketitle
%\author
\begin{section}{Aufgabe}%Aufgabe 1
Es seien die Kurven \(\gamma_1 : t \mapsto t + \imag(\sqrt 3)t, 0 < t < \infty\) 
und 
\(\gamma_2 : t \mapsto \cos t + \imag\sqrt 3 \sin t, 0 \leq t < \pi\)
sowie die Abbildung \(\vfunc{f}{\bC}{\bC}{z}{z^2-z\sqrt 2}\)
 gegeben.
\begin{itemize}
\item[a)]
Berechnen Sie den Schnittpunkt der Kurven $\gamma_1$ und $\gamma_2$ und den Schnittwinkel.\\
Wir finden die Gemeinsames punkt in dem wir das Problem im $\bR^2$ umstellen.
\[\vfunc{\vec{\upgamma}_1}{\bR}{\bR^2}{a}{\begin{pmatrix}\Re(\gamma_1(a))\\\Im(\gamma_1(a))\end{pmatrix}
=\begin{pmatrix}a\\a\sqrt{3}\end{pmatrix}}\]
\[\vfunc{\vec{\upgamma}_2}{\bR}{\bR^2}{b}{\begin{pmatrix}\Re(\gamma_2(b))\\\Im(\gamma_2(b))\end{pmatrix}
=\begin{pmatrix}\cos b\\\sin(b)\sqrt{3}\end{pmatrix}}\]
Wir bauen die folgendes Gleichungsystem:
\[\begin{pmatrix}a\\a\sqrt{3}\end{pmatrix}=\begin{pmatrix}\cos(b)\\\sin(b)\sqrt{3}\end{pmatrix}
\implies a=\cos(b)=\sin(b)\implies b_1=\frac{\pi}{4},\quad b_2=\pi+\frac{\pi}{4}\]
%fuck Ich habe die 2 schnittpunkt auch berechnet weil es mir spät ausgefallen ist, die Rechnung wird kommentiert
Wir wurden 2 Schnittpunkten haben, jedoch fällt 1 raus, da $0 \leq b < \pi$, also, wir haben nur 1 schnittpunkt:
\newcommand{\schnt}[3]{#3_{#1,#2}}
\[\schnt{\gamma_1}{\gamma_2}{P}=\begin{pmatrix}\frac{\sqrt{2}}{2}\\\frac{\sqrt{2}}{2} \sqrt{3}\end{pmatrix}%,\quad \schnt{\gamma_1}{\gamma_2}{2}=-\schnt{\gamma_1}{\gamma_2}{1}
\]
\begin{figure}[H]
\centering
\begin{tikzpicture}[scale=3.5]
    \tkzInit[xmin=-1.5,xmax=1.5,xstep=1,ymin=0,ymax=2,ystep=1]
    \tkzGrid[sub]
    \tkzAxeX[label={}]
    \tkzDrawX[label=$\Re(z)$]
    \tkzAxeY[label={}]
    \tkzDrawY[label=$\Im(z)$]
    %\draw[gray!50, thin, step=0.5] (-1,-3) grid (5,4);
    %\draw[very thick,->] (-1,0) -- (5.2,0) node[right] {$\Re(z)$};
    %\draw[very thick,->] (0,-3) -- (0,4.2) node[above] {$\Im(z)$};
     %\path [draw=blue,postaction={on each segment={mid arrow=red}},domain=0:1,smooth,variable=\t]
      %(.2,0) -- (3,1) plot (\t,\t);
\path[draw,domain=0:1,smooth,variable=\t]plot [smooth,tension=1] (\t,1.732*\t) [arrow inside={end=stealth,opt={black,scale=2}}{0.5}];
    \path[draw,domain=0:1,smooth,variable=\t]
    plot [smooth,tension=1] ({cos((\t r)*pi)},{1.732*sin((\t r)*pi)}) [arrow inside={end=stealth,opt={black,scale=2}}{0.15}];
    %\path[draw,domain=0:1,smooth,variable=\t]plot (\t,1.732*\t);
   % \foreach \x in {-1,...,5} \draw (\x,0.05) -- (\x,-0.05) node[below] {\tiny\x};
    %\foreach \y in {-3,...,4} \draw (-0.05,\y) -- (0.05,\y) node[right] {\tiny\y};

    %\fill[blue!50!cyan,opacity=0.3] (1,-3) -- (1,4) -- (-1,4) -- (-1,-3) -- cycle;

    %\draw (1,-3) -- node[below,sloped] {\tiny$\gamma(t)$} (1,4);
   % \draw (1,-3) -- (3,1) -- node[below left,sloped] {\tiny$2x_1-x_2\leq5$} (4.5,4);
    %\draw (-1,1) -- node[above,sloped] {\tiny$-x_1+2x_2\leq3$} (5,4);

\end{tikzpicture}
\caption{Parametrizierung von $\gamma_1$ und $\gamma_2$}
\end{figure}
Wir berechnen die Schnittwinkel zwischen $\gamma_1$ und $\gamma_2$, dafür linearisieren (Taylor-Approximation) die Funktionen an der Stelle $\schnt{\gamma_1}{\gamma_2}{P}$:\\
\[\vec{\upgamma_1}'(a)=\begin{pmatrix}1\\\sqrt{3}\end{pmatrix}\]
\[\vec{\upgamma_2}'(b)=\begin{pmatrix}-\sin(b)\\\cos(b)\sqrt{3}\end{pmatrix}\]
für $\schnt{\gamma_1}{\gamma_2}{P}$:
\[\phi_1=\arccos(\frac{\begin{psmallmatrix}1\\\sqrt{3}\end{psmallmatrix}\cdot \begin{psmallmatrix}-\sin(b_1)\\\cos(b_1)\sqrt{3}\end{psmallmatrix}}{\left|\begin{psmallmatrix}1\\\sqrt{3}\end{psmallmatrix}\right|\left|\begin{psmallmatrix}-\sin(b_1)\\\cos(b_1)\sqrt{3}\end{psmallmatrix}\right|})
=\arccos(\frac{-\sin(b_1)+3\cos(b_1)}{\sqrt{1+3} \sqrt{3 \cos[2](b_1)+\sin[2](b_1)}})\]
\[=\arccos(\frac{-\frac{\sqrt{2}}{2}+3\frac{\sqrt{2}}{2}}{\sqrt{4} \sqrt{3 (\frac{\sqrt{2}}{2})^2+(\frac{\sqrt{2}}{2})^2}})=
\arccos(\frac{2\frac{\sqrt{2}}{2}}{2 \sqrt{3 \frac{1}{2}+\frac{1}{2}}})
=\arccos(\frac{\sqrt{2}}{2 \sqrt{4\frac{1}{2}}})
=\arccos(\frac{1}{2})=\frac{\pi}{3}\]
%für $\schnt{\gamma_1}{\gamma_2}{2}$ folgt analog:
%\[\phi_2=\arccos(\frac{-\sin(b_2)+3\cos(b_2)}{\sqrt{1+3} \sqrt{3 \cos[2](b_2)+\sin[2](b_2)}})
%=\arccos(\frac{\frac{\sqrt{2}}{2}-3\frac{\sqrt{2}}{2}}{\sqrt{4} \sqrt{3 (\frac{\sqrt{2}}{2})^2+(\frac{\sqrt{2}}{2})^2}})
%=\arccos(-\frac{1}{2})=\frac{2\pi}{3}\]

\item[b)]
Ermitteln Sie Schnittpunkt und Schnittwinkel der Bildkurven $f(\gamma_1)$ und $f(\gamma_2)$.\\
Die Schnittpunkt der Bildkurven von $f(\gamma_1)$ und $f(\gamma_2)$ ist gleich der Bild der schnittpunkt der Kurven, d.h
\[\schnt{f(\gamma_1)}{f(\gamma_2)}{P}=f(\schnt{\gamma_1}{\gamma_2}{P})%,\quad i\in\qty{1,2}
\]
Berechnung:
\[\schnt{f(\gamma_1)}{f(\gamma_2)}{\widetilde{P}}=f\left(\frac{\sqrt{2}}{2}+\imag\left(\frac{\sqrt{6}}{2}\right)\right)=\left(\frac{\sqrt{2}}{2}+\imag\left(\frac{\sqrt{6}}{2}\right)\right)^2-\left(\frac{\sqrt{2}}{2}+\imag\left(\frac{\sqrt{6}}{2}\right)\right)\sqrt{2}\]
\[=-1+\imag \sqrt{3}-1 - \imag \sqrt(3)=-2\]
\[\schnt{f(\gamma_1)}{f(\gamma_2)}{P}=\begin{pmatrix}-2\\0\end{pmatrix}\]
%und
%\[\widetilde{P}_{f(\gamma_1),f(\gamma_2),2}
%=f\left(-\frac{\sqrt{2}}{2}-\imag\left(\frac{\sqrt{6}}{2}\right)\right)
%=\left(\frac{\sqrt{2}}{2}+\imag\left(\frac{\sqrt{6}}{2}\right)\right)^2+\left(\frac{\sqrt{2}}{2}+\imag\left(\frac{\sqrt{6}}{2}\right)\right)\sqrt{2}\]
%\[=-1+\imag \sqrt{3}+1 + \imag \sqrt(3)=2 \imag \sqrt 3\]
%\[\schnt{f(\gamma_1)}{f(\gamma_2)}{2}=\begin{pmatrix}0\\2\sqrt 3\end{pmatrix}\]
\newcommand{\f}{\vec{\text{\fontshape{ui}\selectfont f}}}
Um die Schnittwinkel zu berechnen liniearisieren wir auch $\f(\vec{\upgamma}_i(t)), i \in \qty{1,2}$

Es gilt: \[\frac{\partial}{\partial t}\left(\f(\vec{\upgamma}_i(t))\right)=\dfrac{\partial \f}{\partial \vec{\upgamma_i}}\dfrac{\partial \vec{\upgamma_i}}{\partial t}\]
\(\dfrac{\partial \vec{\upgamma_i}}{\partial t_i}\) sind die Linearisierungen von $\vec{\upgamma_i}$, wir haben diese schon berechnet.
\[\dfrac{\partial \f}{\partial \vec{\upgamma_i}}=
\begin{pmatrix}
    \dfrac{\partial}{\partial \Re \gamma_i}\Re f&& \dfrac{\partial}{\partial \Im \gamma_i}\Re f\\
    \dfrac{\partial}{\partial \Re \gamma_i}\Im f&& \dfrac{\partial}{\partial \Im \gamma_i}\Im f
\end{pmatrix}\]
Also, mit \(x=\Re \gamma_i\) und \(y=\Im \gamma_i\)
\[f(z)=f(x+\imag y)
=(x+\imag y)^2-\sqrt{2} (x+\imag y)
=x^2+2 \imag x y-\sqrt{2} x-y^2-\imag \sqrt{2} y
=\underbrace{x^2-\sqrt{2} x-y^2}_{\Re f}
+\imag\underbrace{\left(2 x y-\sqrt{2} y\right)}_{\Im f}\]
Daraus folgt\[\frac{\partial \f}{\partial \vec{\upgamma_i}}=
\begin{pmatrix}
    2\Re \gamma_i -\sqrt{2}&& -2\Im \gamma_i\\
    2\Im \gamma_i&& 2\Re \gamma_i -\sqrt{2}
\end{pmatrix}\]
für \(\gamma_1\):
\newcommand{\pfa}{\frac{\partial \f}{\partial \vec{\upgamma_1}}\frac{\partial \vec{\upgamma_1}}{\partial a}}
\newcommand{\pfb}{\frac{\partial \f}{\partial \vec{\upgamma_2}}\frac{\partial \vec{\upgamma_2}}{\partial b}}
\[\pfa=
\begin{pmatrix}
    2\Re \gamma_1 -\sqrt{2}&& -2\Im \gamma_1\\
    2\Im \gamma_1&& 2\Re \gamma_1 -\sqrt{2}
\end{pmatrix}
\cdot\begin{pmatrix}1\\1\sqrt{3}\end{pmatrix}
=
\begin{pmatrix}
    2a -\sqrt{2}&& -2a\sqrt{3}\\
    2a\sqrt{3}&& 2a -\sqrt{2}
\end{pmatrix}
\cdot\begin{pmatrix}1\\\sqrt{3}\end{pmatrix}
\]
\[=\begin{pmatrix}
    2a-\sqrt{2}-6a\\
    2a\sqrt{3}+\sqrt{3}(2a -\sqrt{2})
\end{pmatrix}
=\begin{pmatrix}
    -\sqrt{2}-4a\\
    4\sqrt{3}a-\sqrt{6}
\end{pmatrix}\]
für \(\gamma_2\):
\[\pfb=
\begin{pmatrix}
    2\Re \gamma_2 -\sqrt{2}&& -2\Im \gamma_2\\
    2\Im \gamma_2&& 2\Re \gamma_2 -\sqrt{2}
\end{pmatrix}
\cdot\begin{pmatrix}-\sin{b}\\\cos{b}\sqrt{3}\end{pmatrix}
=
\begin{pmatrix}
    2\cos{b} -\sqrt{2}&& -2\sin{b}\sqrt{3}\\
    2\sin{b}\sqrt{3}&& 2\cos{b} -\sqrt{2}
\end{pmatrix}
\cdot\begin{pmatrix}-\sin{b}\\\cos{b}\sqrt{3}\end{pmatrix}
\]
\[=\begin{pmatrix}
    -\sin{b}(2\cos{b}-\sqrt{2})-6\sin{b}\cos{b}\\
    -2\sin^2{b}\sqrt{3}+\cos{b}\sqrt{3}(2\cos{b}-\sqrt{2})
\end{pmatrix}
=\begin{pmatrix}
    \sqrt{2} \sin (b)-8 \sin (b) \cos (b)  \\
    -2 \sqrt{3} \sin ^2(b)+2 \sqrt{3} \cos ^2(b)-\sqrt{6} \cos (b)
\end{pmatrix}\]
Wir berechnen die Skalarprodukt \[\pfa\cdot\pfb=\begin{pmatrix}
    -\sqrt{2}-4a\\
    4\sqrt{3}a-\sqrt{6}
\end{pmatrix}\cdot\begin{pmatrix}
    \sqrt{2} \sin (b)-8 \sin (b) \cos (b)  \\
    -2 \sqrt{3} \sin ^2(b)+2 \sqrt{3} \cos ^2(b)-\sqrt{6} \cos (b)
\end{pmatrix}
\]\[\underbrace{=}_{a=\cos(b)=\sin(b)}
\begin{pmatrix}
    -\sqrt{2}-4a\\
    4\sqrt{3}a-\sqrt{6}
\end{pmatrix}\cdot\begin{pmatrix}
    \sqrt{2} a-8 a^2  \\
    -2 \sqrt{3} a^2+2 \sqrt{3} a^2-\sqrt{6} a
\end{pmatrix}=
\begin{pmatrix}
    -\sqrt{2}-4a\\
    4\sqrt{3}a-\sqrt{6}
\end{pmatrix}\cdot\begin{pmatrix}
    \sqrt{2} a-8 a^2  \\
    -\sqrt{6} a
\end{pmatrix}\]
\[=\left(-4 a-\sqrt{2}\right) \left(\sqrt{2} a-8 a^2\right)-\sqrt{6} a \left(4 \sqrt{3} a-\sqrt{6}\right)
=4 a \left(8 a^2-2 \sqrt{2} a+1\right)\] und die Längen, bzw Norm der liniearisierte Vektoren. 
\[\left|\pfa\right|=\left|\begin{pmatrix}
    -\sqrt{2}-4a\\
    4\sqrt{3}a-\sqrt{6}
\end{pmatrix}\right|=\sqrt{(\sqrt{2}+4a)^2+(4\sqrt{3}a-\sqrt{6})^2}\] 
\[\left|\pfb\right|=\left|\begin{pmatrix}
    \sqrt{2} a-8 a^2  \\
    -\sqrt{6} a
\end{pmatrix}\right|=\sqrt{(\sqrt{2}a-8a^2)^2+(\sqrt{6} a)^2}\] 
Daraus folgt:
\[\cos(\phi_2)=\frac{\pfa\cdot\pfb}{\left|\pfa\right|\left|\pfb\right|}
=\frac{4 a \left(8 a^2-2 \sqrt{2} a+1\right)}{\sqrt{\left(\sqrt{2}+4a\right)^2+\left(4\sqrt{3}a-\sqrt{6}\right)^2}\sqrt{\left(\sqrt{2}a-8a^2\right)^2+\left(\sqrt{6} a\right)^2}}
\]
\[=\frac{a \sqrt{8 a^2-2 \sqrt{2} a+1}}{2 \sqrt{a^2 \left(8 a^2-2 \sqrt{2} a+1\right)}}
=\frac{a \sqrt{8 a^2-2 \sqrt{2} a+1}}{2 \left|a\right| \sqrt{ \left(8 a^2-2 \sqrt{2} a+1\right)}}=\text{sgn}(a)\frac{1}{2}\]
für \(\schnt{f(\gamma_1)}{f(\gamma_2)}{P}\) gilt $a=\frac{\sqrt 2}{2}$:
\[\phi_2=\arccos(\frac{1}{2})=\frac{\pi}{3}\]
%für \(\schnt{f(\gamma_1)}{f(\gamma_2)}{2}\) gilt $a_2=-\frac{\sqrt 2}{2}$:
%\[\phi_2=\arccos(\frac{a \sqrt{8 a_2^2-2 \sqrt{2} a_2+1}}{2 \sqrt{a_2^2 \left(8 a_2^2-2 \sqrt{2} a_2+1\right)}})=\arccos(-\frac{1}{2})=\]
\end{itemize}


\end{section}
\begin{section}{Aufgabe}%Aufgabe 2
Ermitteln Sie fur die Möbius-Transformationen $h_1$ und $h_2$ mit:
\[h_1(z)=\frac{2z-1}{3z-2},\quad h_2(z)=\frac{z-1}{-2z+3}\]
die Komposition $h_1 \circ h_2$ und die inversen Transformationen $h_1^{-1}$ und $h_2^{-1}$.
\\
Die Komposition von $h_1 \circ h_2$ folgt:
\[h_1(h_2(z))=h_1\left(\frac{z-1}{-2z+3}\right)=\frac{2\left(\frac{z-1}{-2z+3}\right)-1}{3\left(\frac{z-1}{-2z+3}\right)-2}=\frac{2\left(\frac{z-1}{-2z+3}\right)-1}{3\left(\frac{z-1}{-2z+3}\right)-2}\frac{-2z+3}{-2z+3}\]
\[\frac{2 (z-1)-(-2z+3)}{3\left(z-1\right)-2(-2z+3)}=\frac{4 z -5}{7 z -9}\]
Für die inverse von $h_1^{-1}$
tauschen wir eine Variabel um, \(\omega \longleftrightarrow z\)
wir lösen es auf $\omega$ um.
\[z=\frac{2\omega -1}{3 \omega -2}\]
\[(3 \omega -2)z=2\omega -1\]
\[3\omega z -2z=2\omega -1\]
\[\frac{1 -2z}{2-3z}=\omega\]
\[\omega=\frac{2z-1}{3z-2}=h_1^{-1}\]
Für die inverse von $h_2^{-1}$ folgt analog:
\[z=\frac{\omega -1}{-2 \omega+3}\]
\[(-2 \omega+3)z=\omega -1\]
\[-2 \omega z+3z=\omega -1\]
\[\frac{1 +3z}{1+2z}=\omega\]
\[\omega=\frac{3z+1}{2z+1}=h_2^{-1}\]
\end{section}
\begin{section}{Aufgabe}%Aufgabe 3
Eine Möbius-Transformation $T$ werde durch die Angaben
\[f(1)=1,\quad f(\imag)=1+\imag,\quad f(-1+\imag)=2+2\imag\]
beschrieben.
\begin{itemize}
\item[a)]
Bestimmen Sie den Term $f(z)$ für $z \in \bC$.\\
Wir verwenden den Ansatz: \(f(z)=\frac{z a+b}{z c+d}\)
Wir haben die folgende Systemgleichung
\[1=\frac{a+b}{c+d}\]
\[1+\imag=\frac{\imag a+b}{\imag c+d}\]
\[2+2\imag=\frac{(-1+\imag)a+b}{(-1+\imag)c+d}\]
Daraus folgt:
\begin{align}
    a+b&=c+d \label{eq:1}\\
    (\imag c+d)(\imag+1)=-c+d+\imag(c+d)&=\imag a+b\label{eq:2}
\end{align}
und
\[((-1+\imag)c+d)(2+2\imag)=-4 c+2 \imag d+2 d=-a+ \imag a +b=((-1+\imag)a+b)\]
mit \eqref{eq:2}
\[-4 c+2 \imag d+2 d=-a -c+d+\imag(c+d)\]
\[-3 c+ \imag d+ d=-a +\imag c\]
Man könnte das in reelle und imaginäre teile trennen, aber es ist möglich mit Rotationen, Translationen und Inversionen, eine Möbiustransform zu bauen, wir beobachten, dass $f(1)=1$ ein Feste punkt ist.
Sei $g(z)=f(z+1)-1$ folgt:
\begin{align*}
g(0)&=0    \\
g(\imag-1)&=\imag \\
g(-2+\imag)&=1+2\imag
\end{align*}
Wir drehen unsere Bild so $h(z)=\imag g(z)$:
\begin{align*}
h(0)&=0    \\
h(\imag-1)&=-1 \\
h(-2+\imag)&=-2+\imag
\end{align*}
Wir erhalten eine neue Gleichungsystem.
\[h(0)=\frac{0 a+b}{0 c+d}=\frac{b}{d}=0\implies b=0\]
\[h(\imag-1)=\frac{(\imag-1)a}{(\imag-1) c+d}=-1\implies a=-c-\frac{d}{\imag-1}\]
\[h(\imag-2)=\frac{(\imag-2)a}{(\imag-2) c+d}=\imag-2\implies a=c(\imag-2)+d\]
Somit:
\begin{align*}
c(\imag-2)+d&=-c-\frac{d}{\imag-1}   \\
c(\imag-1)+d&=-\frac{d}{\imag-1}\\
c(\imag-1)^2+d(\imag-1)&=-d\\
c(\imag-1)^2+d \imag&=0\\
-2 \imag c+d \imag&=0\\
-2 c+d &=0\\
d &=2c
\end{align*}
Wir nehmen o.B.d.A dass $c=1$,
dann $d=2$ und $a=(\imag-2)+2=\imag$.
\[h(z)=\frac{\imag z}{z+2}\]
\[f(z)=g(z-1)+1=-\imag h(z-1)+1=-\imag \frac{\imag (z-1)}{z+1}+1= \frac{(z-1)}{z+1}+1\]
\[f(1)=0+1\]
\[f(\imag)=\frac{(\imag-1)}{\imag+1}+1=\frac{(\imag-1)^2}{\left(\imag+1\right)\left(\imag-1\right)^2}+1=\imag+1\]
\[f(\imag-1)=\frac{(\imag-2)}{\imag}+1=-\imag(\imag-2)+1=2\imag+2\]

\item[b)] Bestimmen Sie Bild und Urbild des unendlich fernen Punkts $\infty$
\[\lim_{z\rightarrow\infty}f(z)=\frac{(z-1)}{z+1}+1=1+1=2\]
Unsere Nenner ist $z+1$, daraus folgt dass $\lim_{z\rightarrow-1} f(z)=\infty$

\item[alt b)]
Bestimmen Sie von der offenen Halbgeraden $\gamma : t \mapsto t \imag, 0 < t < \infty$ rechnerisch
das Bild $f(\gamma)$ und beschreiben Sie es als geometrische Figur.
\[f(\gamma)=1-\frac{1}{\gamma}=1-\frac{1}{t \imag}=1+\frac{\imag}{t}\]
Es ist eine Halberade parallel zu der $\imaginary$ Achse, aber nur auf der positiv maginär Seite. $1+a \imag,\quad a<0<\infty$
oder $x=1,y>0$



\end{itemize}
\end{section}
\end{document}
