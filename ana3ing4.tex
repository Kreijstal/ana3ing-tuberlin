\documentclass{scrartcl}
\usepackage{amsmath}
\usepackage{amsfonts}
\usepackage{tabto}
\usepackage{enumerate}
\usepackage[ngerman]{babel}
%\usepackage[T1]{fontenc}
\usepackage{lmodern}
\usepackage[utf8]{inputenc}
\usepackage{tabularx}
\usepackage[paper=a4paper,left=25mm,right=25mm,top=20mm,bottom=20mm]{geometry}
\usepackage{nccmath}
\usepackage{xcolor}
\usepackage{ragged2e}
\usepackage{graphics}
\usepackage{graphicx}
\usepackage{fancyhdr}
\usepackage{mathtools}  
\usepackage{float}
\usepackage{physics}
\usepackage{subcaption}
%\usepackage{wasysym}
%\newcommand*{\vpointer}{\vcenter{\hbox{\scalebox{2}{\Huge\pointer}}}}
\usepackage[linkcolor=blu,colorlinks=false]{hyperref}

%\usepackage{cfr-lm} %ich erinnere mich nicht wieso war das wichtig
\usepackage{newtxmath}
\usepackage{cleveref}
\renewcommand{\headrulewidth}{0.5pt}
\renewcommand{\footrulewidth}{0.5pt}
\usepackage{microtype}  % Minature font tweaks
\usepackage{cancel}
%\usepackage{cmathbb}
\usepackage{tikz,calc}
\usepackage[german=quotes]{csquotes}
%\DeclareSymbolFont{eulerletters}{U}{zeur}{b}{n}
%\DeclareSymbolFont{cmrot1italics}{OT1}{cmr}{m}{it}
%\DeclareMathSymbol{\imag}{\mathord}{eulerletters}{`i}
%\DeclareMathSymbol{\imag}{\mathord}{cmrot1italics}{`i}
\DeclareSymbolFont{uiletters}{OT1}{cmr}{m}{ui}
\DeclareMathSymbol{\imag}{\mathalpha}{uiletters}{`i}
% Common shortcuts
\def\mbb#1{\mathbb{#1}}
\def\mfk#1{\mathfrak{#1}}
%\def\imag{\mathsf{i}}
\def\bN{\mbb{N}}
\def\bC{\mbb{C}}
\def\bR{\mbb{R}}
\def\bQ{\mbb{Q}}
\def\bZ{\mbb{Z}}
\newcommand{\func}[3]{#1\colon#2\to#3}
\newcommand{\vfunc}[5]{\func{#1}{#2}{#3},\quad#4\longmapsto#5}
\newcommand{\floor}[1]{\left\lfloor#1\right\rfloor}
\newcommand{\ceil}[1]{\left\lceil#1\right\rceil}
%Abstand zwischen Absätzen, Zeilenabstände
\voffset26pt 
\parskip6pt
%\parindent1cm  %Rückt erste Zeile eines neuen Absatzes ein
\usepackage{setspace}
\onehalfspacing
\title{Analysis III für Ingenieurwissenschaften}
\author{Juan Pardo Martin (397882)\and Tuan Kiet Nguyen (404029)\and Leonardo Nerini (414193)}
\let\*\cdot
\let\-\rightarrow
\let\_\Rightarrow
\let\>\Leftrightarrow
\definecolor{l}{rgb}{0.0, 0.5, 0.0}
\let\c\textcolor
%\mathbb{R} 
%\begin{align*}
\titlehead
{
\begin{tabular}{ll}
\begin{minipage}{0.5\textwidth}
%\begin{figure}[H]
% \raggedright
 %\includegraphics[scale=0.04]{tu-logo}\\
%\end{figure}
\end{minipage}
\begin{minipage}{0.5\textwidth}
\begin{figure}[H]
\raggedleft
%\includegraphics[scale=0.04]{tu-logo}\\
\end{figure}
\end{minipage}
\end{tabular}\\
\\
  \small
      {
    Ana III Hausaufgabe, 4 Woche\\
    Tutor:  David Sering  \\
	SS 2021
    }


}
\begin{document}
\maketitle
%\author
\begin{section}{Aufgabe}%Aufgabe 1
Ermitteln Sie das Bild der Kurve $\gamma : t \mapsto 1 + \imag t$ unter der Abbildung $f(z) = \frac 1 z$
auf zwei Weisen:
\begin{itemize}
\item[a)]  mit dem auf den Folien \texttt{W4V1S2} und \texttt{W4V1S3} der Vorlesung der 4. Woche verwendeten Verfahren (\enquote{$\omega, \overline{\omega}$})
\item[b)] mit Hilfe der Kreistreue von Möbius-Transformationen. (Gefundene Kreisgleichungen sind zu beweisen, z.B. anhand einer Uberlegung oder durch Nachrechnen.) 
\end{itemize}
\begin{subsection}{Antwort}
\begin{itemize}
\item[a)]  die Kurve $\gamma$ lässt sich darstellen als einen \enquote{Kreis} (Im sinne von einen Riemannschekugel) als 
\begin{alignat*}{2}
    \Re(z)&=1\quad\quad\quad\quad &|&z\>\frac 1 \omega\\
    \Re(\frac {1}{ \omega})&=1 &|&\Re(z)\>\frac 1 2 \left(z+\overline{z}\right)\\
    \frac 1 2 \left(\frac {1}{ \omega}+\overline{\frac {1}{ \omega}}\right)&=1 &|&\omega\*\overline{\omega}\\
    \frac 1 2 \left(\frac {1}{ \omega}+\overline{\left(\frac {1}{ \omega}\right)}\right)\*\omega\*\overline{\omega}&=\omega\*\overline{\omega} &|&\overline{\left(\frac {1}{ \omega}\right)}\> \frac{1}{\overline{\omega}}\\
   \frac 1 2 \left(\frac {1}{ \omega}+\frac{1}{\overline{\omega}}\right)\*\omega\*\overline{\omega}&=\omega\*\overline{\omega} \\
  \frac 1 2 \left(\overline{\omega}+\omega \right)&=\omega\*\overline{\omega} &|&-\frac 1 2 \left(\overline{\omega}+\omega \right)\\
  \omega\*\overline{\omega}-\frac 1 2 \left(\overline{\omega}+\omega \right)&=0 &|&+\frac{1}{4}\\
  \omega\*\overline{\omega}-\frac{1}{2}\overline{\omega}-\frac{1}{2}\omega+\frac{1}{4}&=\frac{1}{4}&|&\omega\*\overline{\omega}-\frac{1}{2}\overline{\omega}-\frac{1}{2}\omega+\frac{1}{4}\>\left(\omega-\frac{1}{2}\right)\left(\overline{\omega}-\frac{1}{2}\right)\\
  \left(\omega-\frac{1}{2}\right)\left(\overline{\omega}-\frac{1}{2}\right)&=\frac{1}{4}\\
  \left|\omega-\frac{1}{2}\right|^2&=\frac{1}{4} &|&\sqrt{\cdots}\\
  \left|\omega-\frac{1}{2}\right|&=\frac{1}{2}
\end{alignat*}
Der Kreis \(\Re(z)=1\) wird auf den Kreis \(\left|z-\frac{1}{2}\right|=\frac{1}{2}\) aufgebildet.
\item[b)] Da $\frac{1}{z}$ ist eine Inversion, und damit eine Mobiustransformation. Wir nehmen 3 beliebige $t$, wie $\left(0,1,2\right)$, die in $\gamma$ zu $\left(1,1+\imag,1+2\imag\right)$ abgebildet sind. Diese punkten werden durch $f$ in $\left(1,\frac{1-\imag}{2},\frac{1-2\imag}{5}\right)$
Wir wissen dass diese Punkten einen Kreis abbilden. Wir können versuche der Zentrum und Radius zu finden so:
\[\left|z-c\right|=r,\] wobei $c\in \bC$ und $r \in \bR^+$ ist für die Folgende Gleichungsystem
\begin{align*}
    \left|1-c\right|&=r\\
    \left|\frac{1-\imag}{2}-c\right|&=r\\
    \left|\frac{1-2\imag}{5}-c\right|&=r
\end{align*}
oder
\begin{align}
    (1-\Re(c))^2+\Im(c)^2&=r^2 \label{eq:1}\\
    \left(\frac{1}{2}-\Re(c)\right)^2+\left(-\frac{1}{2}-\Im(c)\right)^2&=r^2\label{eq:2}\\
    \left(\frac{1}{5}-\Re(c)\right)^2+\left(-\frac{2}{5}-\Im(c)\right)^2&=r^2\label{eq:3}
\end{align}
Wir beobachten, dass \eqref{eq:2} und \eqref{eq:3} gleich \eqref{eq:1} sind.
\begin{align*}
    \left(\frac{1}{2}-\Re(c)\right)^2+\left(\frac{1}{2}+\Im(c)\right)^2&=(1-\Re(c))^2+\Im(c)^2\\
    \left(\frac{1}{5}-\Re(c)\right)^2+\left(\frac{2}{5}+\Im(c)\right)^2&=(1-\Re(c))^2+\Im(c)^2
\end{align*}
Wir multiplizieren aus.
\begin{align*}
    \cancel{\Im(c)^2}+\Im(c)+\cancel{\Re(c)^2}-\Re(c)+\frac{1}{2}&=\cancel{\Im(c)^2+\Re(c)^2}-2 \Re(c)+1\\
    \cancel{\Im(c)^2}+\frac{4 \Im(c)}{5}+\cancel{\Re(c)^2}-\frac{2 \Re(c)}{5}+\frac{1}{5}&=\cancel{\Im(c)^2+\Re(c)^2}-2 \Re(c)+1\\
    \Im(x)+\Re(c)=\frac{1}{2}\\
    \frac{4 \Im(c)}{5}+\frac{8 \Re(c)}{5}=\frac{4}{5}
\end{align*}
oder
\begin{align*}
    \begin{pmatrix}
    1&&1\\\frac{8}{5}&&\frac{4}{5}
\end{pmatrix}\*\begin{pmatrix}
    \Re(c)\\\Im(c)
\end{pmatrix}&=\begin{pmatrix}
    \frac{1}{2}\\\frac{4}{5}
\end{pmatrix}\\
  \begin{pmatrix}
    \Re(c)\\\Im(c)
\end{pmatrix}&=-\frac{5}{4}\begin{pmatrix}
    \frac{4}{5}&&-1\\
    -\frac{8}{5}&&1
\end{pmatrix}\*\begin{pmatrix}
    \frac{1}{2}\\\frac{4}{5}
\end{pmatrix}\\
\begin{pmatrix}
    \Re(c)\\\Im(c)
\end{pmatrix}&=\begin{pmatrix}
    \frac{1}{2}\\0
\end{pmatrix}
\end{align*}
Also $c=\frac{1}{2}$, jetzt um $r$ zu finden. Wir ersetzen diese Ergebnis in \eqref{eq:1} und somit ist $r=\frac{1}{2}$
Also wir haben den Kreis $\left|z-\frac{1}{2}\right|=\frac{1}{2}$
Wir wollen nun herausfinden welche die Durchlaufsinn, wo würde Re(z)<1 im oder Außerkreis abgebildet. Die Richtung lässt sich bestimmen in den man $\partial_t f(\gamma(t))$ an einen Punkt berechnet
\begin{align*}
        \partial_t \frac{1-\imag t}{1+t^2}
=-\frac{2 (1-\imag t) t}{\left(t^2+1\right)^2}-\frac{\imag}{t^2+1}\\
=-\frac{2 t}{\left(t^2+1\right)^2}+\imag \left(\frac{2
   t^2}{\left(t^2+1\right)^2}-\frac{1}{t^2+1}\right)\quad &|t\rightarrow 0\\
   =-\imag
\end{align*}
An der Punkt $(1,0)$ entspricht dass eine Durchrichtung im Uhrsinn, das bedeutet dass $Re(z)<2$ wird Außerhalb dieses Kreis abgebildet.\\
\begin{figure}[H]
\centering
\begin{subfigure}{.47\textwidth}
  \centering
  \begin{tikzpicture}[x=0.8cm,y=0.8cm]

    \draw[gray!50, thin, step=0.5] (-1,-3) grid (5,4);
    \draw[very thick,->] (-1,0) -- (5.2,0) node[right] {$\Re(z)$};
    \draw[very thick,->] (0,-3) -- (0,4.2) node[above] {$\Im(z)$};

    \foreach \x in {-1,...,5} \draw (\x,0.05) -- (\x,-0.05) node[below] {\tiny\x};
    \foreach \y in {-3,...,4} \draw (-0.05,\y) -- (0.05,\y) node[right] {\tiny\y};

    \fill[blue!50!cyan,opacity=0.3] (1,-3) -- (1,4) -- (-1,4) -- (-1,-3) -- cycle;

    \draw (1,-3) -- node[below,sloped] {\tiny$\gamma(t)$} (1,4);
   % \draw (1,-3) -- (3,1) -- node[below left,sloped] {\tiny$2x_1-x_2\leq5$} (4.5,4);
    %\draw (-1,1) -- node[above,sloped] {\tiny$-x_1+2x_2\leq3$} (5,4);

\end{tikzpicture}
  \caption{Kurve vor $f$}
\end{subfigure}%
{\LARGE$\xrightarrow{\frac{1}{z}}$}%
\begin{subfigure}{.47\textwidth}
  \centering
  \begin{tikzpicture}[x=2cm,y=2cm]

    \draw[gray!50, thin, step=0.25] (-1,-1-0.5) grid (2,1+0.5);
    \draw[very thick,->] (-1,0) -- (2.1,0) node[right] {$\Re(z)$};
    \draw[very thick,->] (0,-1.5) -- (0,1.6) node[above] {$\Im(z)$};

    \foreach \x in {-1,...,2} \draw (\x,0.05) -- (\x,-0.05) node[below] {\tiny\x};
    \foreach \y in {-1,...,1} \draw (-0.05,\y) -- (0.05,\y) node[right] {\tiny\y};

    \fill[blue!50!cyan,opacity=0.3, even odd rule] (-1,-1.5) rectangle (2,1.5) (1/2,0) circle (1/2);

    \draw (1/2,0) circle  (1/2) node[below,sloped,yshift=-1] {\tiny$f(\gamma(t))$};
   % \draw (1,-3) -- (3,1) -- node[below left,sloped] {\tiny$2x_1-x_2\leq5$} (4.5,4);
    %\draw (-1,1) -- node[above,sloped] {\tiny$-x_1+2x_2\leq3$} (5,4);

\end{tikzpicture}
  \caption{Kurve nach $f$}
\end{subfigure}
\caption{Möbiustransform}
\end{figure}





\end{itemize}
\end{subsection}
\end{section}
\begin{section}{Aufgabe}%Aufgabe 2
    Bestimmen Sie eine Möbius-Transformation $f$, welche erstens die Eigenschaft $f(1 + \imag) = 1 - \imag$
hat und zweitens die von links nach rechts durchlaufene reelle Achse $\rightarrow$ mit der von unten nach oben durchlaufenen imaginären Achse $\uparrow$ vertauscht und außerdem den im
Ursprung zentrierten und im positiven Drehsinn durchlaufenen Einheitskreis $\circlearrowleft$ in denselben Kreis $\circlearrowright$, aber mit negativem Durchlaufssinn, uberführt: 
\[f(\rightarrow)=\uparrow,\quad f(\uparrow)=\rightarrow,\quad f(\circlearrowleft)=\circlearrowright\]
\end{section}
\begin{subsection}{Antwort}
    Wir wissen dass die Inverse $\frac{1}{z}$ bildet die Einheitskreis in denselben Kreis aber mit anderen Durchlaufsinn. Außerdem Rotationen sind Multiplikation mit werte der Einheitskreis, wenn wir $\imag$ multiplizieren Rotieren wir der Kreis um 90 grad, wir können diese Funktionen komponieren:
    $f(z)=z\*\imag\circ \frac{1}{z}=\frac{\imag}{z}$
    Wir prüfen.
    \[f(1)=\imag,\quad f(\imag)=1,\quad f(1+\imag)=\frac{\imag}{1+\imag}=\frac{1}{1-\imag}\]
    ICH GLAUBE DASS DIE AUFGABE FALSCH IST. DA DER KREIS SCHON BESTIMMT IST, UND KANN SICH NICHT BEWEGEN VON URPSRUNG (KEINE TRANSLATION)
\end{subsection}
\begin{section}{Aufgabe}%Aufgabe 3
Gegeben sei das Randwertproblem
\begin{alignat*}{3}
 \Delta u&=0,\quad &\text{für }& y>0\text{ und }x^2+(y-4)^2>1,\\
u(x,y)&=0,\quad &\text{für }& y=0,\\
 u(x,y)&=1,\quad &\text{für }& x^2+(y-4)^2=1
\end{alignat*}
Lösen Sie es mit der Methode der harmonischen Verpflanzung, indem Sie als Verpflanzungsabbildung die Funktion
\[T(z)=\frac{z-\imag \sqrt{15}}{z+\imag \sqrt{15}}\]
und als Ansatzfunktion die harmonische Funktion $\ln(x^2+ y^2)$ benutzen.
\begin{subsection}{Antwort}
    Sei $\vfunc{\text{ReIm}}{\bC}{\bR^2}{x+\imag y}{\begin{pmatrix}x\\y\end{pmatrix}}$
    Wir trennen die imaginären und reellen Teilen von $T$.
    \[
    T(x+\imag y)
    =\frac{(x+\imag y)-\imag \sqrt{15}}{(x+\imag y)+\imag \sqrt{15}}
    =\frac{x^2+y^2-15-\imag 2\sqrt{15}}{x^2+(\sqrt{15}+y)^2}
    \]
    \[\text{ReIm}(T(x+\imag y))=\frac{1}{x^2+(\sqrt{15}+y)^2}
    \begin{pmatrix}
        x^2+y^2-15\\-2\sqrt{15}
    \end{pmatrix}
    \]
    Wir finden heraus wo sind die Rände abgebildet.\\
    Wir nehmen 3 Punkten, für Kreis $\Im(z)=0$.\\
    zum Beispiel, Punkte $\left(0,1,\infty\right)$, sind in $\left(-1,\frac{1-\sqrt{15}}{1+\sqrt{15}},1\right)$ abgebildet.\\
    Wir simplifizieren:
    \[\frac{1-\sqrt{15}}{1+\sqrt{15}}=\frac{1}{16}(1-\imag\sqrt{15})^2=\frac{1}{8}\left(7-\imag\sqrt{15}\right).\] Also die Einheitskreis, wo $\Im(z)$ zu $|z|>1$ abgebildet wird.\\
    Wir nehmen wieder 3 punkten für Kreis $|z-\imag 4|>1$.\\
    Z.b $\left(\imag 5,\imag 3,1+\imag 4\right)$.\\
    Also
    \begin{align*}
        T(1+\imag 4)&=\frac{(1+4 \imag)-\imag \sqrt{15}}{(1+4 \imag)+\imag \sqrt{15}}=-1+\frac{8-2 i}{\sqrt{15}+(4-i)}\\
        T(\imag 5)&=\frac{5 \imag-\imag \sqrt{15}}{5 \imag+\imag \sqrt{15}}=\frac{5-\sqrt{15}}{5 +\sqrt{15}}=4-\sqrt{15}\\
        T(\imag 3)&=\frac{3 \imag-\imag \sqrt{15}}{3 \imag+\imag \sqrt{15}}=\frac{3-\sqrt{15}}{3 +\sqrt{15}}=-4+\sqrt{15}\\
    \end{align*}
    Wir finden die imaginären und reellen Teilen von $T(1+\imag 4)$ und finden wir ihre Betrag
    \[T(1+\imag 4)
    =-1+\frac{8-2 i}{\sqrt{15}+(4-i)}
    =-1+\frac{2+8(4+\sqrt{15})}{1+\left(4+\sqrt{15}\right)^2}+\imag\left(\frac{8-2\left(4+\sqrt{15}\right)}{1+\left(4+\sqrt{15}\right)^2}\right)
    =\frac{1-\imag\sqrt{15}}{4(4+\sqrt{15})}
    \]
    \[ =1-\frac{\sqrt{15}}{4}+\imag\left(\frac{15}{4}-\sqrt{15}    \right)\]
    \[\left|T(1+\imag 4)\right|=\sqrt{\left(1-\frac{15}{4}\right)^2+ \left(\frac{15}{4}-\sqrt{15} \right)^2}
    =\sqrt{\frac{31}{16}-\sqrt{15}{2}+\frac{465}{16}-\frac{15 \sqrt{15}}{2}}
    =\sqrt{31-8 \sqrt{15}}=4-\sqrt{15}\]
    Wir können deutlich sehen, dass alle die Punkten haben betrag $4-\sqrt{15}$, das bedeutet, sie bilden einen Kreis im Ursprung mit einen Radius von $4-\sqrt{15}$
    Wir schreiben unsere Randwertproblem um.
    \begin{alignat*}{3}
        \Delta U(\xi,\chi)&=0, \quad &\text{für }& 31-8 \sqrt{15}<\xi^2+\chi^2<1\\
        U(\xi,\chi)&=0, \quad &\text{für }& \xi^2+\chi^2=1\\
        U(\xi,\chi)&=1, \quad &\text{für }& \xi^2+\chi^2=31-8 \sqrt{15}
    \end{alignat*}
    Wir verwenden den Ansatz $U=A\ln(\xi^2+\chi^2)+B$
    \begin{alignat*}{5}
        0&=A\ln(\xi^2+\chi^2)+B&=&A\ln(1)+B&\implies B&=0\\
        1&=A\ln(\xi^2+\chi^2)&=&A\ln(31-8 \sqrt{15})& \implies A&=\frac{1}{\ln(31-8 \sqrt{15})}\\
    \end{alignat*}
    \[\therefore U(\xi,\chi)=\frac{1}{\ln(31-8 \sqrt{15})}\ln(\xi^2+\chi^2)\]
    Wir finden $u(x,y)$ folgendes: 
    \begin{align*}
        u(x,y)=U(\text{ReIm}(T(x+\imag y)))\\
        u(x,y)=U(\frac{1}{x^2+(\sqrt{15}+y)^2}\left(x^2+y^2-15\right),\frac{1}{x^2+(\sqrt{15}+y)^2}\left(-2\sqrt{15}\right))\\
        u(x,y)=\frac{1}{\ln(31-8 \sqrt{15})}\ln((\xi)^2+\chi^2)
    \end{align*}
\end{subsection}
\end{section}    
\end{document}
