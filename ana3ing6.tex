\documentclass{scrartcl}
\def\wochenumber{6}
\usepackage{amsmath}
\usepackage{amsfonts}
\usepackage{tabto}
\usepackage{enumerate}
\usepackage[ngerman]{babel}
%\usepackage[T1]{fontenc}
\usepackage{lmodern}
\usepackage[utf8]{inputenc}
\usepackage{tabularx}
\usepackage[paper=a4paper,left=25mm,right=25mm,top=20mm,bottom=20mm]{geometry}
\usepackage{nccmath}
\usepackage{xcolor}
\usepackage{ragged2e}
\usepackage{graphics}
\usepackage{graphicx}
\usepackage{fancyhdr}
\usepackage{mathtools}  
\usepackage{float}
\usepackage{physics}
\usepackage{subcaption}
%\usepackage{wasysym}
%\newcommand*{\vpointer}{\vcenter{\hbox{\scalebox{2}{\Huge\pointer}}}}
\usepackage[linkcolor=blu,colorlinks=false]{hyperref}

%\usepackage{cfr-lm} %ich erinnere mich nicht wieso war das wichtig
\usepackage{newtxmath}
\usepackage{cleveref}
\renewcommand{\headrulewidth}{0.5pt}
\renewcommand{\footrulewidth}{0.5pt}
\usepackage{microtype}  % Minature font tweaks
\usepackage{cancel}
%\usepackage{cmathbb}
\usepackage{tikz,calc}
\usepackage{tkz-fct}  
\usepackage{tkz-euclide}
\usepackage{derivative}
\usepackage{ifthen}
\usetikzlibrary{decorations.pathreplacing,decorations.markings}
%\usetikzlibrary{decorations.markings}
\tikzset{
    set arrow inside/.code={\pgfqkeys{/tikz/arrow inside}{#1}},
    set arrow inside={end/.initial=>, opt/.initial=},
    /pgf/decoration/Mark/.style={
        mark/.expanded=at position #1 with
        {
            \noexpand\arrow[\pgfkeysvalueof{/tikz/arrow inside/opt}]{\pgfkeysvalueof{/tikz/arrow inside/end}}
        }
    },
    arrow inside/.style 2 args={
        set arrow inside={#1},
        postaction={
            decorate,decoration={
                markings,Mark/.list={#2}
            }
        }
    },
}
\tikzset{
  % style to apply some styles to each segment of a path
  on each segment/.style={
    decorate,
    decoration={
      show path construction,
      moveto code={},
      lineto code={
        \path [#1]
        (\tikzinputsegmentfirst) -- (\tikzinputsegmentlast);
      },
      curveto code={
        \path [#1] (\tikzinputsegmentfirst)
        .. controls
        (\tikzinputsegmentsupporta) and (\tikzinputsegmentsupportb)
        ..
        (\tikzinputsegmentlast);
      },
      closepath code={
        \path [#1]
        (\tikzinputsegmentfirst) -- (\tikzinputsegmentlast);
      },
    },
  },
  % style to add an arrow in the middle of a path
  mid arrow/.style={postaction={decorate,decoration={
        markings,
        mark=at position .5 with {\arrow[#1]{stealth}}
      }}},
}

\usepackage[german=quotes]{csquotes}
%\DeclareSymbolFont{eulerletters}{U}{zeur}{b}{n}
%\DeclareSymbolFont{cmrot1italics}{OT1}{cmr}{m}{it}
%\DeclareMathSymbol{\imag}{\mathord}{eulerletters}{`i}
%\DeclareMathSymbol{\imag}{\mathord}{cmrot1italics}{`i}
\DeclareSymbolFont{uiletters}{OT1}{cmr}{m}{ui}
\DeclareMathSymbol{\imag}{\mathalpha}{uiletters}{`i}
% Common shortcuts
\def\mbb#1{\mathbb{#1}}
\def\mfk#1{\mathfrak{#1}}
%\def\imag{\mathsf{i}}
\def\bN{\mbb{N}}
\def\bC{\mbb{C}}
\def\bR{\mbb{R}}
\def\bQ{\mbb{Q}}
\def\bZ{\mbb{Z}}
\newcommand{\func}[3]{#1\colon#2\to#3}
\newcommand{\vfunc}[5]{\func{#1}{#2}{#3},\quad#4\longmapsto#5}
\newcommand{\floor}[1]{\left\lfloor#1\right\rfloor}
\newcommand{\ceil}[1]{\left\lceil#1\right\rceil}
%Abstand zwischen Absätzen, Zeilenabstände
\voffset26pt 
\parskip6pt
%\parindent1cm  %Rückt erste Zeile eines neuen Absatzes ein
\usepackage{setspace}
\onehalfspacing
\title{Analysis III für Ingenieurwissenschaften}
\author{Juan Pardo Martin (397882)\and Tuan Kiet Nguyen (404029)\and Leonardo Nerini (414193)}
\let\*\cdot
\let\-\rightarrow
\let\_\Rightarrow
\let\>\Leftrightarrow
\definecolor{l}{rgb}{0.0, 0.5, 0.0}
\newcommand\ReIm[0]{\text{ReIm}}
\makeatletter
\newcommand\sqrtnorm[1]{
    \count@=0
    \sqrt{
    \@for\levar:=#1\do{
        \advance\count@ 1
        \ifnum\count@=1
        \else +
        \fi
        {\levar}^2
        }
    }
}
\makeatother
\let\c\textcolor
%\mathbb{R} 
%\begin{align*}
\titlehead
{
\begin{tabular}{ll}
\begin{minipage}{0.5\textwidth}
%\begin{figure}[H]
% \raggedright
 %\includegraphics[scale=0.04]{tu-logo}\\
%\end{figure}
\end{minipage}
\begin{minipage}{0.5\textwidth}
\begin{figure}[H]
\raggedleft
%\includegraphics[scale=0.04]{tu-logo}\\
\end{figure}
\end{minipage}
\end{tabular}\\
\\
  \small
      {
    Ana III Hausaufgabe, \wochenumber{} Woche\\
    Tutor:  David Sering  \\
	SS 2021
    }


}
%\usepackage{mathtools} 
\begin{document}
\maketitle
%\author
\begin{section}{Aufgabe}%Aufgabe 1
Berechnen Sie mit dem Integralsatz oder der Integralformel von Cauchy die Integrale
\[\int_{|z-1|=3}\frac{\cos \pi z} {z^2 - 7z + 10}dz\text{, und }\int_{|z|=1}\tan z dz\]
und begrunden Sie Ihr Vorgehen. Benutzen Sie ohne Beweis, dass die Nüllstellen der
Cosinusfunktion alle reell sind.

\begin{subsection}{Antwort}
    \begin{itemize}
        \item[a)]
    Wir können das Integral verschieben:
    \[\int_{|z+1-1|=3} \frac{\cos (\pi  (z+1))}{(z+1)^2-7 (z+1)+10} \, dz\]
    und den Nenner reduzieren.
    \[\int_{|z|=3} \frac{-\cos (\pi  (z))}{(-4+z) (-1+z)} \, dz\]
Sei $G=\qty{z\in\bC\mid \qty|z|<3+\varepsilon,\varepsilon>0}$ mit eine beliebig klein $\varepsilon$, eine offene Gebiet ist die Punkte wo $|z|=3$ eine Kompakte Teilmenge von $G$. Man kann sehen dass 
\[f(z)=-\frac{\cos(\pi z)}{-4+z}\] ist auf G analytisch, da ihre Pole ist außer den Kreis $3+\varepsilon$.
Daraus folgt
\[\int_{|z|=3} \frac{-\cos (\pi  (z))}{(-4+z) (-1+z)} \, dz=2\pi \imag f(1)=-2\pi \imag \frac{\cos(\pi)}{-3}=-\frac{2}{3}\pi \imag \]
\item[b)] 
Die polen sind die Nullstelen von $\cos(z)$, die sind alle Reelle zahlen $\frac{\pi}2 +\pi n,n \in \bZ $
Keine nullstelle ist im Einheitskreis, also die Integral ergibt 0.
    \end{itemize} 
\end{subsection}
\end{section}
\begin{section}{Aufgabe}%2
Gegeben ist die Funktion $\vfunc{u}{\bR^2}{\bR}{(x,y)}{2xy+x^2-y^2}$. Ermitteln Sie Stellen
und Werte der globalen Extrema der Funktion $u$ auf der abgeschlossenen Kreisscheibe $\qty{(x, y) \in \bR^2\mid x^2 + y^2 \leq 1}.$
Benutzen Sie bei Bedarf die speziellen Werte  \[ \sin \frac{\pi}{8}=\frac{1}{2} \sqrt{2-\sqrt{2}} \text { und } \cos \frac{\pi}{8}=\frac{1}{2} \sqrt{2+\sqrt{2}} \]
    \begin{subsection}{Antwort}
  Wir wissen dass die Extrema der Funktionen sind in Rand, wir verwenden der Lagrange verfahren um Extrema zu finden Kritische punkten:
  \[\nabla_{x,y} u = - \lambda \nabla_{x,y} g\]
  wobei $g(x,y)=x^2+y^2-1=0$,
  Also 
  \[\Lambda(x,y,\lambda) = u(x,y) + \lambda \cdot g(x,y)\]
  \[\nabla_{x,y} u=\begin{pmatrix}
      2 x + 2 y\\
      2 x - 2 y
  \end{pmatrix}\]
  und
    \[\nabla_{x,y} g=\begin{pmatrix}
      2 x \\
      2 y
  \end{pmatrix}\]
   Wir können berechnen \[det\begin{pmatrix}
 2 x+2 y & 2 x \\
 2 x-2 y & 2 y \\
\end{pmatrix}=0=-4 x^2+8 x y+4 y^2=-4 \qty(x^2-2 x y-y^2)
\]
Wir können sehen dass wegen $g$ liegt $x$ und $y$ auf der Einheitskreis. Also
\[\qty(\cos(\varphi)^2-2 \cos(\varphi) \sin(\varphi)-\sin(\varphi)^2)=0\]
Wir verwenden den Hinweis und setzten wir $\varphi=\frac{\pi}{8}$
to be continued..

Alternativ können wir berechnen \[det\begin{pmatrix}
 2 \lambda +2 & 2 \\
 2 & 2 \lambda -2 \\
\end{pmatrix}=0=(2 \lambda +2)(2 \lambda -2)-4=4 \lambda ^2-8\implies \lambda = \pm \sqrt{2}
\]
für $\lambda =\pm \sqrt{2}$
\[2x+2y\pm 2\sqrt{2} x =0 \implies x\qty(2\pm 2 \sqrt{2})=-2 y \implies y=-x\qty(1\pm \sqrt{2})\]
Also
\[x^2+x^2\qty(1\pm \sqrt{2})^2-1=0\implies x^2\qty(1+\qty(3\pm 2\sqrt{2}))=1\]
\[x=\pm\frac{1}{\sqrt{4\pm 2\sqrt{2}}}=\]
Also für $\lambda = \sqrt{2}$
\[x=\pm \frac{\sqrt{2+\sqrt{2}}}{2}\] 
und \[y=-x\qty(1+ \sqrt{2})\]
und für $\lambda = - \sqrt{2}$
\[x=\pm \frac{\sqrt{2-\sqrt{2}}}{2}\]
und \[y=-x\qty(1- \sqrt{2})\]
\newcommand{\xone}{\frac{\sqrt{2+\sqrt{2}}}{2}}
\newcommand{\xtwo}{\frac{\sqrt{2-\sqrt{2}}}{2}}
\newcommand{\yone}{\xone\qty(1+ \sqrt{2})}
\newcommand{\ytwo}{\xtwo\qty(1- \sqrt{2})}
Wir haben hier 4 Kritischepunkten
\begin{align*}
  \qty(\xone,-\yone),\\
  \qty(-\xone,\yone),\\
  \qty(\xtwo,-\ytwo),\\
  \qty(-\xtwo,\ytwo)  
\end{align*}
Wir evaluieren auf diesen Stellen.
\begin{align*}
    u\qty(\xone,-\yone)=u\qty(-\xone,\yone)\\
    =\frac{1}{4} \left(2-\sqrt{2}\right)-\frac{1}{2} \left(1-\sqrt{2}\right)
   \left(2-\sqrt{2}\right)-\frac{1}{4} \left(1-\sqrt{2}\right)^2 \left(2-\sqrt{2}\right)=-4 + 3 \sqrt{2}\\
   u\qty(\xtwo,-\ytwo)=u\qty(-\xtwo,\ytwo)\\
   =-4 - 3 \sqrt{2}
\end{align*}

%Wir können sehen dass wegen $g$ liegt $x$ und $y$ auf der Einheitskreis. Also
%\[\qty(\cos(\varphi)^2-2 \cos(\varphi) \sin(\varphi)-\sin(\varphi)^2)=0\]
    \end{subsection}
\end{section}
\begin{section}{Aufgabe}
Mit dieser Aufgabe wollen wir ein Ergebnis mit zwei verschiedenen Methoden herleiten,
nämlich mit der Methode \enquote{direkt} und mit der Methode \enquote{Poissonsche Integralformel}.
Auf einer gewissen Kreisscheibe im \( \mathbb{R}^{2} \) sei das Randwertproblem (RWP) für eine Funktion \( u: \mathbb{R}^{2} \rightarrow \mathbb{R} \) gegeben 
\[ \begin{array}{ll} \Delta u=0 & \text { für } x^{2}+y^{2}<1, \\ u(x, y)=4 x^{2}-2 & \text { für } x^{2}+y^{2}=1 \end{array} \] 
\begin{itemize}
\item[a)]Lösen Sie dieses RWP mit einer geeigneten Ansatzfunktion. 
\item[b)] (Methode \enquote{direkt}:) Bestimmen Sie anhand der in a) erhaltenen Lösung \( u \) den Wert \( u\left(\frac{4}{5}, 0\right) \). 
\item[c)] Stellen Sie die Poissonsche Integralformel zur Berechnung des Werts \( u\left(\frac{4}{5}, 0\right) \) auf (ähnlich wie in einer Tutoriumsaufgabe) 
\item[d)] Schreiben Sie das Integral in Ihrer Integralformel in ein komplexes Integral über den Einheitskreis \( |z|=1 \) um (ähnlich wie in einer Tutoriumsaufgabe) 
\item[e)]  Werten Sie Ihr komplexes Integral mit Hilfe der Cauchyschen Integralsätze und Ergebnissen aus der Vorlesung aus. (Methode, \enquote{Poissonsche Integralformel}) Hierbei dürfen Sie ohne Nachweis folgende Partialbruchzerlegung benutzen: \[ \frac{1}{z^{2}\left(z-\frac{4}{5}\right)\left(z+\frac{5}{4}\right)}=\frac{41}{20} \cdot \frac{1}{z}+\frac{1}{z^{2}}+\frac{64}{25} \cdot 
\frac{1}{z-\frac{5}{4}}-\frac{125}{36} \cdot \frac{1}{z-\frac{4}{5}}\]
Begründen Sie aber, warum Sie diese Identität benutzen. Der Rechenweg ist ansonsten (wie immer) vollständig darzustellen.
\end{itemize}


      \begin{subsection}{Antwort}
\begin{itemize}
\item[a)]
wir nehmen die Folgende Ansatz 
\[u\qty(x,y)=4x^2-2+A(x^2+y^2-1)\]
wenn $A=-2$ ist dann $u(x,y)=2x^2-2y^2\implies \Delta u =0$
\item[b)] 
\[u(\frac{4}{5},0)=2\qty(\frac{4}{5})^2\] 
\item[c)]  
\[u\qty(r \mathrm{e}^{\imag \theta})=\frac{1}{2\pi}\int_0^{2\pi} u\qty(R \mathrm{e}^{\imag \phi}) \frac{R^2-r^2}{R^2 -2R r \cos(\theta-\phi)+r^2}\dd \phi. \]
Unsere große Radius ist 1, kleine Radius und $\theta$ ist Betrag und Arg von $(\frac{4}{5},0)$. $\theta=0;r=\frac{4}{5}$
\[u\qty(\frac{4}{5})=\frac{1}{2\pi}\int_0^{2\pi} u\qty(\mathrm{e}^{\imag \phi}) \frac{1-r^2}{1 -2 r \cos(-\phi)+r^2}\dd \phi. \]
\[u\qty(\frac{4}{5})=\frac{1}{2\pi}\int_0^{2\pi} u\qty(\mathrm{e}^{\imag \phi}) \frac{1-\qty(\frac{4}{5})^2}{1 -2 \qty(\frac{4}{5})^2 \cos(-\phi)+\qty(\frac{4}{5})^2}\dd \phi. \]
\[u\qty(\frac{4}{5};0)=\frac{1}{2\pi}\int_0^{2\pi}2\qty(\cos[2](\phi)-\sin[2](\phi))\frac{9}{41-40 \cos (\phi )}\dd\phi \]
\[u\qty(\frac{4}{5};0)=\frac{1}{2\pi}\int_0^{2\pi}2\qty(\cos[2](\phi)-\sin[2](\phi))\frac{9}{41-40 \cos (\phi )} \dd\phi\]
\[u\qty(\frac{4}{5};0)=\frac{1}{2\pi}\int_0^{2\pi}2\qty(2\cos[2](\phi)-1)\frac{9}{41-40 \cos (\phi )}\dd\phi \]

\item[d)] 
Wir verwenden $\cos(\phi)=\qty(\frac{1}{2}\qty(\mathrm{e}^{\imag \phi}+\mathrm{e}^{-\imag \phi}))$ 
\[u\qty(\frac{4}{5};0)=\frac{1}{2\pi}\int_0^{2\pi}2\qty(2\qty(\frac{1}{2}\qty(\mathrm{e}^{\imag \phi}+\mathrm{e}^{-\imag \phi}))^2-1)\frac{- \imag 9 \mathrm{e}^{-\imag \phi}}{41-40 \qty(\frac{1}{2}\qty(\mathrm{e}^{\imag \phi}+\mathrm{e}^{-\imag \phi}))} \imag \mathrm{e}^{\imag \phi}\dd\phi \]
\[u\qty(\frac{4}{5};0)=\frac{1}{2\pi}\int_0^{2\pi}2\qty(\frac{1}{2}\qty(\mathrm{e}^{\imag \phi}+\mathrm{e}^{-\imag \phi})^2-1)\frac{- \imag 9 \mathrm{e}^{-\imag \phi}}{41-40 \qty(\frac{1}{2}\qty(\mathrm{e}^{\imag \phi}+\mathrm{e}^{-\imag \phi}))} \imag \mathrm{e}^{\imag \phi}\dd\phi \]
\[u\qty(\frac{4}{5};0)=\frac{1}{2\pi}\int_{\qty|z|=1}\qty(\qty(z+z^{-1})^2-2)\frac{- \imag 9 z^{-1}}{41-20 \qty(\qty(z+z^{-1}))} \dd z \]
\[u\qty(\frac{4}{5};0)=\frac{1}{2\pi}\int_{\qty|z|=1}\qty(\qty(z+z^{-1})^2-2)\frac{- \imag 9 z^{-1}}{41-20 \qty(z+z^{-1})} \dd z \]
\[u\qty(\frac{4}{5};0)=\frac{1}{2\pi}\int_{\qty|z|=1}\qty(\qty(z+z^{-1})^2-2)\frac{- \imag 9 z^{-1}}{41-20 \qty(z+z^{-1})} \dd z \]
\item[e)] 
\end{itemize}
    \end{subsection}  
\end{section}
\end{document}
