\documentclass{scrartcl}
\def\wochenumber{7}
\usepackage{amsmath}
\usepackage{amsfonts}
\usepackage{tabto}
\usepackage{enumerate}
\usepackage[ngerman]{babel}
%\usepackage[T1]{fontenc}
\usepackage{lmodern}
\usepackage[utf8]{inputenc}
\usepackage{tabularx}
\usepackage[paper=a4paper,left=25mm,right=25mm,top=20mm,bottom=20mm]{geometry}
\usepackage{nccmath}
\usepackage{xcolor}
\usepackage{ragged2e}
\usepackage{graphics}
\usepackage{graphicx}
\usepackage{fancyhdr}
\usepackage{mathtools}  
\usepackage{float}
\usepackage{physics}
\usepackage{subcaption}
%\usepackage{wasysym}
%\newcommand*{\vpointer}{\vcenter{\hbox{\scalebox{2}{\Huge\pointer}}}}
\usepackage[linkcolor=blu,colorlinks=false]{hyperref}

%\usepackage{cfr-lm} %ich erinnere mich nicht wieso war das wichtig
\usepackage{newtxmath}
\usepackage{cleveref}
\renewcommand{\headrulewidth}{0.5pt}
\renewcommand{\footrulewidth}{0.5pt}
\usepackage{microtype}  % Minature font tweaks
\usepackage{cancel}
%\usepackage{cmathbb}
\usepackage{tikz,calc}
\usepackage{tkz-fct}  
\usepackage{tkz-euclide}
\usepackage{derivative}
\usepackage{ifthen}
\usetikzlibrary{decorations.pathreplacing,decorations.markings}
%\usetikzlibrary{decorations.markings}
\tikzset{
    set arrow inside/.code={\pgfqkeys{/tikz/arrow inside}{#1}},
    set arrow inside={end/.initial=>, opt/.initial=},
    /pgf/decoration/Mark/.style={
        mark/.expanded=at position #1 with
        {
            \noexpand\arrow[\pgfkeysvalueof{/tikz/arrow inside/opt}]{\pgfkeysvalueof{/tikz/arrow inside/end}}
        }
    },
    arrow inside/.style 2 args={
        set arrow inside={#1},
        postaction={
            decorate,decoration={
                markings,Mark/.list={#2}
            }
        }
    },
}
\tikzset{
  % style to apply some styles to each segment of a path
  on each segment/.style={
    decorate,
    decoration={
      show path construction,
      moveto code={},
      lineto code={
        \path [#1]
        (\tikzinputsegmentfirst) -- (\tikzinputsegmentlast);
      },
      curveto code={
        \path [#1] (\tikzinputsegmentfirst)
        .. controls
        (\tikzinputsegmentsupporta) and (\tikzinputsegmentsupportb)
        ..
        (\tikzinputsegmentlast);
      },
      closepath code={
        \path [#1]
        (\tikzinputsegmentfirst) -- (\tikzinputsegmentlast);
      },
    },
  },
  % style to add an arrow in the middle of a path
  mid arrow/.style={postaction={decorate,decoration={
        markings,
        mark=at position .5 with {\arrow[#1]{stealth}}
      }}},
}

\usepackage[german=quotes]{csquotes}
%\DeclareSymbolFont{eulerletters}{U}{zeur}{b}{n}
%\DeclareSymbolFont{cmrot1italics}{OT1}{cmr}{m}{it}
%\DeclareMathSymbol{\imag}{\mathord}{eulerletters}{`i}
%\DeclareMathSymbol{\imag}{\mathord}{cmrot1italics}{`i}
\DeclareSymbolFont{uiletters}{OT1}{cmr}{m}{ui}
\DeclareMathSymbol{\imag}{\mathalpha}{uiletters}{`i}
% Common shortcuts
\def\mbb#1{\mathbb{#1}}
\def\mfk#1{\mathfrak{#1}}
%\def\imag{\mathsf{i}}
\def\bN{\mbb{N}}
\def\bC{\mbb{C}}
\def\bR{\mbb{R}}
\def\bQ{\mbb{Q}}
\def\bZ{\mbb{Z}}
\newcommand{\func}[3]{#1\colon#2\to#3}
\newcommand{\vfunc}[5]{\func{#1}{#2}{#3},\quad#4\longmapsto#5}
\newcommand{\floor}[1]{\left\lfloor#1\right\rfloor}
\newcommand{\ceil}[1]{\left\lceil#1\right\rceil}
%Abstand zwischen Absätzen, Zeilenabstände
\voffset26pt 
\parskip6pt
%\parindent1cm  %Rückt erste Zeile eines neuen Absatzes ein
\usepackage{setspace}
\onehalfspacing
\title{Analysis III für Ingenieurwissenschaften}
\author{Juan Pardo Martin (397882)\and Tuan Kiet Nguyen (404029)\and Leonardo Nerini (414193)}
\let\*\cdot
\let\-\rightarrow
\let\_\Rightarrow
\let\>\Leftrightarrow
\definecolor{l}{rgb}{0.0, 0.5, 0.0}
\newcommand\ReIm[0]{\text{ReIm}}
\makeatletter
\newcommand\sqrtnorm[1]{
    \count@=0
    \sqrt{
    \@for\levar:=#1\do{
        \advance\count@ 1
        \ifnum\count@=1
        \else +
        \fi
        {\levar}^2
        }
    }
}
\makeatother
\let\c\textcolor
%\mathbb{R} 
%\begin{align*}
\titlehead
{
\begin{tabular}{ll}
\begin{minipage}{0.5\textwidth}
%\begin{figure}[H]
% \raggedright
 %\includegraphics[scale=0.04]{tu-logo}\\
%\end{figure}
\end{minipage}
\begin{minipage}{0.5\textwidth}
\begin{figure}[H]
\raggedleft
%\includegraphics[scale=0.04]{tu-logo}\\
\end{figure}
\end{minipage}
\end{tabular}\\
\\
  \small
      {
    Ana III Hausaufgabe, \wochenumber{} Woche\\
    Tutor:  David Sering  \\
	SS 2021
    }


}
%\usepackage{mathtools} 

\begin{document}
\maketitle
%\author
\begin{section}{Aufgabe}%Aufgabe 1
    Entwickeln Sie die reelle Funktion \( \frac{1}{1+x^{2}} \) in eine reelle Potenzreihe mit Entwicklungspunkt 1. Hierbei ist auch die maximale Konvergenzkreisscheibe zu ermitteln. Hinweise: Sie dürfen diese Aufgabe mit Hilfe von komplexen Zahlen lösen. Die verlangte reelle Potenzreihe darf aber keine komplexe Zahlen mehr enthalten. Reelle Winkelfunktionen brauchen nicht ausgewertet zu werden.
 
\begin{subsection}{Antwort}
Wir haben
\[\frac{1}{1+x^2}=\frac{1}{1-(-x^2)}=\sum_{n=0}^\infty (-x^{2n}), \abs{-x^2}<1,\] aber dann ist die Entwicklungspunkt 0, wi wollen unsere "Polynom-Basis" von $x$ zu $x-1$ wechseln.
Die Singularität ist $x=\pm \imag$

\[x^2+1=a (x-1)^2+b(x-1)+c = a x^2-2ax+a+bx-b+c\]
Mit Koeffizientenvergleich
\[a=1;-2+b=0\>b=2;1-2+c=1\>c=2\]

\[\frac{1}{1+x^2}=\frac{1}{2+2(x-1)+(x-1)^2}=\frac{1}{2}\frac{1}{1+(x-1)+\frac{1}{2}(x-1)^2}\]
\[=\frac{1}{2}\frac{1}{1-\qty(-\qty((x-1)+\frac{1}{2}(x-1)^2))}\]
Geometrische Reihenfolge:
\[=\frac{1}{2}\sum_{n=0}^{\infty}\qty(-\qty((x-1)+\frac{1}{2}(x-1)^2))^n, \qty|-\qty((x-1)+\frac{1}{2}(x-1)^2)|<1\]

Wir verwenden andere Methode:
\[\qty(\frac{1}{4}+\frac{\imag}{4})\qty(-\imag\qty(-\frac{1}{2}-\frac{\imag}{2})^n+\qty(-\frac{1}{2}+\frac{\imag}{2})^n)
=\frac{1}{2\sqrt{2}}\ee^{\imag\frac{\pi}{4}}\qty(\ee^{\imag\frac{3\pi}{2}}\frac{1}{\sqrt{2}}\ee^{-\imag \frac{3\pi}{4}}+\ee^{\imag \frac{3\pi}{4}})\]
\[=\qty(2^{-\qty(1+\frac{n}{2})}\qty(\cos(\frac{3 n \pi}{4})-\sin(\frac{3 n \pi}{4})))\]
\[\sum_{n=0}^{\infty}\qty(2^{-\qty(1+\frac{n}{2})}\qty(\cos(\frac{3 n \pi}{4})-\sin(\frac{3 n \pi}{4})))(x-1)^n\]
und ihre Konvergenzradius ist $\sqrt{2}$

\end{subsection}
\end{section}
\begin{section}{Aufgabe}%2
    Entwickeln Sie die Funktion \( \log z \) in eine Potenzreihe mit dem Entwicklungspunkt $\imag$. (Vergessen Sie nicht, den Konvergenzbereich anzugeben. Ermitteln Sie den Wert der Reihe \[ \sum_{n=1} \frac{\imag^{n}}{n} \] 
    \begin{subsection}{Antwort}
 
    \end{subsection}
\end{section}
\begin{section}{Aufgabe}
    Berechnen Sie die Integrale \[ \text { a) } \int_{|z-1|=2} \frac{\mathrm{e}^{\imag \pi z}}{(z-1)^{4}} \dd z, \quad \text { b) } \quad \int_{|z-\pi|=1} \frac{z^{4} \sin z}{(z-\pi)^{6}} \dd z \] Tipp: Benutzen Sie gerne die Leibnizsche Produktregel.
      \begin{subsection}{Antwort}

    \end{subsection}  
\end{section}
\end{document}
