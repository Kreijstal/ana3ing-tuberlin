\documentclass{scrartcl}
\usepackage{amsmath}
\usepackage{amsfonts}
\usepackage{tabto}
\usepackage{enumerate}
\usepackage[ngerman]{babel}
\usepackage[T1]{fontenc}
\usepackage{lmodern}
\usepackage[utf8]{inputenc}
\usepackage{tabularx}
\usepackage[paper=a4paper,left=25mm,right=25mm,top=20mm,bottom=20mm]{geometry}
\usepackage{nccmath}
\usepackage{xcolor}
\usepackage{ragged2e}
%\usepackage{graphics}
\usepackage{graphicx}
\usepackage{fancyhdr}
\usepackage{mathtools}  
\usepackage{float}
\usepackage{physics}
\usepackage[linkcolor=blu,colorlinks=false]{hyperref}
\usepackage{cleveref}
%\usepackage{breqn} breqn doesnt work wtf
\usepackage{newtxmath}
\renewcommand{\headrulewidth}{0.5pt}
\renewcommand{\footrulewidth}{0.5pt}
\usepackage{microtype}  % Minature font tweaks

% Common shortcuts
\def\mbb#1{\mathbb{#1}}
\def\mfk#1{\mathfrak{#1}}

\def\bN{\mbb{N}}
\def\bC{\mbb{C}}
\def\bR{\mbb{R}}
\def\bQ{\mbb{Q}}
\def\bZ{\mbb{Z}}
\newcommand{\func}[3]{#1\colon#2\to#3}
\newcommand{\vfunc}[5]{\func{#1}{#2}{#3},\quad#4\longmapsto#5}
\newcommand{\floor}[1]{\left\lfloor#1\right\rfloor}
\newcommand{\ceil}[1]{\left\lceil#1\right\rceil}
%Abstand zwischen Absätzen, Zeilenabstände
\voffset26pt 
\parskip6pt
%\parindent1cm  %Rückt erste Zeile eines neuen Absatzes ein
\usepackage{setspace}
\onehalfspacing
\title{Analysis III für Ingenieurwissenschaften}
\author{Juan Pardo Martin (397882)\and Tuan Kiet Nguyen (404029)\and Leonardo Nerini (414193)}
\let\*\cdot
\let\-\rightarrow
\let\_\Rightarrow
\let\>\Leftrightarrow
\definecolor{l}{rgb}{0.0, 0.5, 0.0}
\let\c\textcolor
%\mathbb{R} 
%\begin{align*}
\titlehead
{
\begin{tabular}{ll}
\begin{minipage}{0.5\textwidth}
%\begin{figure}[H]
% \raggedright
 %\includegraphics[scale=0.04]{tu-logo}\\
%\end{figure}
\end{minipage}
\begin{minipage}{0.5\textwidth}
\begin{figure}[H]
\raggedleft
%\includegraphics[scale=0.04]{tu-logo}\\
\end{figure}
\end{minipage}
\end{tabular}\\
\\
  \small
      {
    Ana III Hausaufgabe, 2 Woche\\
    Tutor:  David Sering  \\
	SS 2021
    }


}
\begin{document}

\maketitle
%\author

\begin{section}{Aufgabe}%Aufgabe 1
Beweisen Sie mit Hilfe der Definition der komplexen Potenz die in der Vorlesung angegebenen Formeln
\begin{center}
a) \((z^p)'=p z^{p-1}\),    b) \((a^z)'=a^z \log(a)\)
\end{center}

und geben Sie dabei an, für welche komplexen Werte von $z$, $p$ und $a$ diese Formeln
sinnvoll sind.
\begin{itemize}
\item[a)]
\[z^p=e^{p \log(z)}\]
Wir wissen dass
\[\log(z)'=\frac{\partial u}{\partial x}+i \frac{\partial v}{\partial x}=\frac{x-y}{r}=\frac{z \overline{z}}{z}=\frac{1}{z}\]
und \[f(g(z))'=f'(g(z))g'(z)\]
Also \[(e^{p \log(z)})'=e^{p \log(z)}\frac{p}{z}=p\frac{e^{p \log(z)}}{z}=p\frac{z^p}{z}=p z^{p-1}\]
In diesem fall \(p\in \bC;z\in \mathbb G\subset \mathbb C\setminus\qty{0}\)
\item[b)]
\newpage
\[a^z=e^{z \log(a)}\]
und
\[(e^{z \log(a)})'=e^{z \log(a)}\log(a)=a^z \log(a)\]
In diesem fall \(z\in \bC;a\in \mathbb G\subset \mathbb C\setminus\qty{0}\)
\end{itemize}
\end{section}
\begin{section}{Aufgabe}%Aufgabe 2
Gegeben seien von einer analytischen Funktion \(f(z)\) die Eigenschaften \(f(i) = i\) sowie
\[\Re f(x + iy) = y \sin x \sinh y + x \cos x \cosh y,\ x, y \in \bR.\]
Bestimmen Sie diese Funktion \(f(z)\).


Da $f$ analytisch ist, wissen wir dass:
\[\Delta u(x,y)=\Delta v(x,y)=0\]
Wir sagen
\[u(x,y)=y \sin x \sinh y + x \cos x \cosh y\]
\[\partial_x u=y \cos x \sinh y + \cosh y (\cos x-x \sin x )\]
\[\partial_x^2 u=-y \sin x \sinh y + \cosh y (-\sin x-(\sin x+x \cos x) )\]
\[\partial_y u=\sin x(\sinh y + y \cosh y)+x \cos x \sinh y\]
\[\partial_y^2 u=\sin x(\cosh y + (y \sinh y+\cosh y))+x \cos x \cosh y\]
\(\Delta u=0\), $u$ ist deshalb Harmonisch
Sodass $f$ analytisch seien kann, dann $f$ muss überall in seine Definitionsbereich differenzierbar sein.
Die C-R-DGL sollen erfüllt sein:
\begin{align}
\partial_x u &= \partial_y v = y \cos x \sinh y + \cosh y (\cos x-x \sin x )  \label{crdgl1}\\ 
\partial_y u &= -\partial_x v = (\sin x(\sinh y + y \cosh y)+x \cos x \sinh y) \label{crdgl2}
\end{align}
Da wir \(\partial_y v\) und \(\partial_x v\) haben, können wir nach Harmonizität in $v$ prüfen.
\[\partial_y^2 v=\cos x(\sinh y +y \cosh y)+\sinh y (\cos x-x \sin x )\]
\[\partial_x^2 v=-(\cos x(\sinh y + y \cosh y)+ \sinh y(\cos x-x \sin x ))\]
\(\Delta v=0\), $v$ ist deshalb Harmonisch.
Und jetzt müssen wir ein $v(x,y)$ finden sodass \eqref{crdgl1} und \eqref{crdgl2} gelten.
Wir integrieren
\[\int y e^y dy=e^y y -\int e^y dy = e^y y - e^y \]
\[\int y e^{-y} dy=-e^{-y} y +\int e^{-y} dy = -e^{-y} y - e^{-y} \]
Daraus folgt:
\[\int y  \sinh y dy=\frac{1}{2} \int y(e^y-e^{-y}) dy=\frac{1}{2} (y(e^y+e^{-y})-e^y+e^{-y})=(y(\cosh y)-\sinh{y})\]
Somit kann man rechnen:
\begin{multline*}
v(x,y) = 
\int \partial_y v dy=\int y \cos x \sinh y + \cosh y (\cos x-x \sin x ) dy\\ =
\cos x \int y  \sinh y dy + \sinh y (\cosh y (\cos x-x \sin x ) )\\
=\cos x (y(\cosh y)-\sinh{y}) + \sinh y (\cos x-x \sin x ) + C(x)
=\cos x (y\cosh y)-\sinh y (x \sin x ) + C(x)
\end{multline*}
Wir leiten nach $x$ ab um $C(x)$ herauszufinden.
\begin{multline*}
\partial_x v= -\sin(x)(y\cosh y)-\sinh y (\sin x+x\cos x)+C'(x)\overset{!}{=}
-(\sin x(\sinh y + y \cosh y)+x \cos x \sinh y)
\end{multline*}
Dafür muss $C'(x)=0$ bzw Konstant sein.
Wir evaluieren $v(x,y)$ an der stelle $(0,1)$
\[v(0,1)= (\cosh 1)+C\]
Wir nehmen $C=1-\cosh 1$
Somit ist \[f(x+iy)=y \sin (x) \sinh (y)+x \cos (x) \cosh (y)+i (y \cos (x) \cosh (y)-x \sin (x) \sinh (y)+1-\cosh 1)\]
\end{section}
\begin{section}{Aufgabe}%Aufgabe 3
Gegeben sei der Viertelkreisring $G$ mit
\[G:=\qty{(x,y)\in\bR^2 |x,y>0 \land 0<\arctan\frac{y}{x}<\frac{\pi}2 \land 1<x^2+y^2<4}\]
Finden Sie eine Funktion $\func{u}{\overline{G}}{\bR}$, die das Randwertproblem

\begin{alignat*}{2}
\Delta u(x,y)&=0 \hspace{2em}    & \text{für } (x,y)\in G\\
u(x,0)&=0 \hspace{2em}   & \text{für } 1\leq x \leq 2\\
u(2 \cos(t),2\sin(t))&=t \ln (2) \hspace{2em}   & \text{für } 0\leq t \leq \frac\pi 2\\
u(0,y)&=\frac \pi 2 \ln y \hspace{2em}   & \text{für } 1\leq x \leq 2\\
u(\cos(t),\sin(t))&=0 \hspace{2em}   & \text{für } 0\leq t \leq \frac\pi 2\\
\end{alignat*}
löst, indem Sie die {\em Methode der harmonischen Verpflanzung} mit der Verpflanzungsabbildung $f(z) = \log z$ verwenden.\\
\\
Wir wissen das (in $\overline G$).
\[f(x+i y)=\ln(x+iy)=\log\sqrt{x^2+y^2}+i \arctan(\frac y x)\]
Da $x\geq 0 \land y\geq 0$\\
Wir können die Rände berechnen (in $\overline G$). (Mit eine Abbildung $\bC \rightarrow \bR^2$)
\[(x,0)\rightarrow (\ln |x|,0)=(\ln x,0)\]
\[(0,y)\rightarrow (\ln y,\frac{\pi}{2})\] 
\[(a \cos(t),a \sin(t))\rightarrow (\ln |a|,t)\]
und somit folgt:
\begin{alignat*}{3}
u(x,0)&=0  &=&U(\ln x,0)\\
u(0,y)&=\frac \pi 2 \ln y  &=&U(\ln y,\frac \pi 2)\\
u(\cos(t),\sin(t)) &=0 &=&U(0,t)\\
u(2\cos(t),2\sin(t))&=t \ln 2  &=&U(\ln 2,t)
\end{alignat*}
Wir schreiben die übersetzes Randwertproblem.
\begin{alignat*}{3}
U(\xi,0)&=0 \hspace{2em}   & \text{für }& 0\leq \xi \leq \ln 2\\
U(0,\eta)&=0 \hspace{2em}   & \text{für }& 0\leq \eta \leq \frac \pi 2\\
U(\ln 2,\eta)&=\eta \ln 2 \hspace{2em}   & \text{für }& 0\leq \eta \leq \frac \pi 2\\
U(\xi,\frac \pi 2)&=\frac \pi 2 \xi \hspace{2em}   & \text{für }& 0\leq \xi \leq \ln 2
\end{alignat*}
Eine mögliche abbildung könnte sein:
\[U(\xi,\eta)=\xi \eta \]
Daraus folgt
\[u(x,y)=U(\ln\sqrt{x^2+y^2},\arctan(\frac y x))=\ln\sqrt{x^2+y^2}\arctan(\frac y x)\]

Da $\Delta U =0$ und $f=\log z$ analytisch, folgt daraus, dass $u$ Harmonisch sein muss. Also $\Delta u=0$.
\end{section}
\end{document}
