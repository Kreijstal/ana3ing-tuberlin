\documentclass{scrartcl}
\usepackage{amsmath}
\usepackage{amsfonts}
\usepackage{tabto}
\usepackage{enumerate}
\usepackage[ngerman]{babel}
\usepackage[T1]{fontenc}
\usepackage{lmodern}
\usepackage[utf8]{inputenc}
\usepackage{tabularx}
\usepackage[paper=a4paper,left=25mm,right=25mm,top=20mm,bottom=20mm]{geometry}
\usepackage{nccmath}
\usepackage{xcolor}
\usepackage{ragged2e}
%\usepackage{graphics}
\usepackage{graphicx}
\usepackage{fancyhdr}
\usepackage{mathtools}  
\usepackage{float}
\usepackage{physics}
%\usepackage[linkcolor=blu,colorlinks=false]{hyperref}
\usepackage{cleveref}
%\usepackage{breqn}
\usepackage{cfr-lm}
\usepackage{newtxmath}
\renewcommand{\headrulewidth}{0.5pt}
\renewcommand{\footrulewidth}{0.5pt}
\usepackage{microtype}  % Minature font tweaks
%\usepackage{cmathbb}
\DeclareSymbolFont{eulerletters}{U}{zeur}{b}{n}
\DeclareMathSymbol{\imag}{\mathord}{eulerletters}{`i}
% Common shortcuts
\def\mbb#1{\mathbb{#1}}
\def\mfk#1{\mathfrak{#1}}
%\def\imag{\mathsf{i}}
\def\bN{\mbb{N}}
\def\bC{\mbb{C}}
\def\bR{\mbb{R}}
\def\bQ{\mbb{Q}}
\def\bZ{\mbb{Z}}
\newcommand{\func}[3]{#1\colon#2\to#3}
\newcommand{\vfunc}[5]{\func{#1}{#2}{#3},\quad#4\longmapsto#5}
\newcommand{\floor}[1]{\left\lfloor#1\right\rfloor}
\newcommand{\ceil}[1]{\left\lceil#1\right\rceil}
%Abstand zwischen Absätzen, Zeilenabstände
\voffset26pt 
\parskip6pt
%\parindent1cm  %Rückt erste Zeile eines neuen Absatzes ein
\usepackage{setspace}
\onehalfspacing
\title{Analysis III für Ingenieurwissenschaften}
\author{Juan Pardo Martin (397882)\and Tuan Kiet Nguyen (404029)\and Leonardo Nerini (414193)}
\let\*\cdot
\let\-\rightarrow
\let\_\Rightarrow
\let\>\Leftrightarrow
\definecolor{l}{rgb}{0.0, 0.5, 0.0}
\let\c\textcolor
%\mathbb{R} 
%\begin{align*}
\titlehead
{
\begin{tabular}{ll}
\begin{minipage}{0.5\textwidth}
%\begin{figure}[H]
% \raggedright
 %\includegraphics[scale=0.04]{tu-logo}\\
%\end{figure}
\end{minipage}
\begin{minipage}{0.5\textwidth}
\begin{figure}[H]
\raggedleft
%\includegraphics[scale=0.04]{tu-logo}\\
\end{figure}
\end{minipage}
\end{tabular}\\
\\
  \small
      {
    Ana III Hausaufgabe, 3 Woche\\
    Tutor:  David Sering  \\
	SS 2021
    }


}
\begin{document}

\maketitle
%\author

\begin{section}{Aufgabe}%Aufgabe 1
Es seien die Kurven \(\gamma_1 : t \mapsto t + \imag(\sqrt 3)t, 0 < t < \infty\) 
und 
\(\gamma_2 : t \mapsto \cos t + \imag\sqrt 3 \sin t, 0 \leq t < \pi\)
sowie die Abbildung \(\vfunc{f}{\bC}{\bC}{z}{z^2-z\sqrt 2}\)
 gegeben.
\begin{itemize}
\item[a)]
Berechnen Sie den Schnittpunkt der Kurven $\gamma_1$ und $\gamma_2$ und den Schnittwinkel.\\
Wir finden die Gemeinsames punkt in dem wir das Problem im $\bR^2$ umstellen.
\[\vec{\upgamma}_1(a)=\begin{pmatrix}\Re(\gamma_1(a))\\\Im(\gamma_1(a))\end{pmatrix},\quad a\in \bR\]
\item[b)]
Ermitteln Sie Schnittpunkt und Schnittwinkel der Bildkurven $f(\gamma_1)$ und $f(\gamma_2)$.
\end{itemize}


\end{section}
\begin{section}{Aufgabe}%Aufgabe 2
Ermitteln Sie fur die Möbius-Transformationen $h_1$ und $h_2$ mit:
\[h_1(z)=\frac{2z-1}{3z-2},\quad h_2(z)=\frac{z-1}{-2z+3}\]
die Komposition $h_1 \circ h_2$ und die inversen Transformationen $h_1^{-1}$ und $h_2^{-1}$.

\end{section}
\begin{section}{Aufgabe}%Aufgabe 3
Eine Möbius-Transformation $T$ werde durch die Angaben
\[f(1)=1,\quad f(\imag)=1+\imag,\quad f(-1+\imag)=2+2\imag\]
beschrieben.
\begin{itemize}
\item[a)]
Bestimmen Sie den Term $f(z)$ für $z \in \bC$.
\item[b)]
%\setmathfont{Latin Modern Math}

Bestimmen Sie von der offenen Halbgeraden $\gamma : t \mapsto t \imag, 0 < t < \infty$ rechnerisch
das Bild $f(\gamma)$ und beschreiben Sie es als geometrische Figur.
\end{itemize}
\end{section}
\end{document}
