\documentclass{scrartcl}
\usepackage{amsmath}
\usepackage{amsfonts}
\usepackage{tabto}
\usepackage{enumerate}
\usepackage[ngerman]{babel}
\usepackage[T1]{fontenc}
\usepackage{lmodern}
\usepackage[utf8]{inputenc}
\usepackage{tabularx}
\usepackage[paper=a4paper,left=25mm,right=25mm,top=20mm,bottom=20mm]{geometry}
\usepackage{nccmath}
\usepackage{xcolor}
\usepackage{ragged2e}
%\usepackage{graphics}
\usepackage{graphicx}
\usepackage{fancyhdr}
\usepackage{mathtools}  
\usepackage{float}
\usepackage{physics}
%\usepackage[linkcolor=blu,colorlinks=false]{hyperref}
\usepackage{cleveref}
%\usepackage{breqn}
\usepackage{cfr-lm}
\usepackage{newtxmath}
\renewcommand{\headrulewidth}{0.5pt}
\renewcommand{\footrulewidth}{0.5pt}
\usepackage{microtype}  % Minature font tweaks
%\usepackage{cmathbb}
%\DeclareSymbolFont{eulerletters}{U}{zeur}{b}{n}
%\DeclareSymbolFont{cmrot1italics}{OT1}{cmr}{m}{it}
%\DeclareMathSymbol{\imag}{\mathord}{eulerletters}{`i}
%\DeclareMathSymbol{\imag}{\mathord}{cmrot1italics}{`i}
\DeclareSymbolFont{uiletters}{OT1}{cmr}{m}{ui}
\DeclareMathSymbol{\imag}{\mathalpha}{uiletters}{`i}
% Common shortcuts
\def\mbb#1{\mathbb{#1}}
\def\mfk#1{\mathfrak{#1}}
%\def\imag{\mathsf{i}}
\def\bN{\mbb{N}}
\def\bC{\mbb{C}}
\def\bR{\mbb{R}}
\def\bQ{\mbb{Q}}
\def\bZ{\mbb{Z}}
\newcommand{\func}[3]{#1\colon#2\to#3}
\newcommand{\vfunc}[5]{\func{#1}{#2}{#3},\quad#4\longmapsto#5}
\newcommand{\floor}[1]{\left\lfloor#1\right\rfloor}
\newcommand{\ceil}[1]{\left\lceil#1\right\rceil}
%Abstand zwischen Absätzen, Zeilenabstände
\voffset26pt 
\parskip6pt
%\parindent1cm  %Rückt erste Zeile eines neuen Absatzes ein
\usepackage{setspace}
\onehalfspacing
\title{Analysis III für Ingenieurwissenschaften}
\author{Juan Pardo Martin (397882)\and Tuan Kiet Nguyen (404029)\and Leonardo Nerini (414193)}
\let\*\cdot
\let\-\rightarrow
\let\_\Rightarrow
\let\>\Leftrightarrow
\definecolor{l}{rgb}{0.0, 0.5, 0.0}
\let\c\textcolor
%\mathbb{R} 
%\begin{align*}
\titlehead
{
\begin{tabular}{ll}
\begin{minipage}{0.5\textwidth}
%\begin{figure}[H]
% \raggedright
 %\includegraphics[scale=0.04]{tu-logo}\\
%\end{figure}
\end{minipage}
\begin{minipage}{0.5\textwidth}
\begin{figure}[H]
\raggedleft
%\includegraphics[scale=0.04]{tu-logo}\\
\end{figure}
\end{minipage}
\end{tabular}\\
\\
  \small
      {
    Ana III Hausaufgabe, 3 Woche\\
    Tutor:  David Sering  \\
	SS 2021
    }


}
\begin{document}
\maketitle
%\author
\begin{section}{Aufgabe}%Aufgabe 1
Es seien die Kurven \(\gamma_1 : t \mapsto t + \imag(\sqrt 3)t, 0 < t < \infty\) 
und 
\(\gamma_2 : t \mapsto \cos t + \imag\sqrt 3 \sin t, 0 \leq t < \pi\)
sowie die Abbildung \(\vfunc{f}{\bC}{\bC}{z}{z^2-z\sqrt 2}\)
 gegeben.
\begin{itemize}
\item[a)]
Berechnen Sie den Schnittpunkt der Kurven $\gamma_1$ und $\gamma_2$ und den Schnittwinkel.\\
Wir finden die Gemeinsames punkt in dem wir das Problem im $\bR^2$ umstellen.
\[\vfunc{\vec{\upgamma}_1}{\bR}{\bR^2}{a}{\begin{pmatrix}\Re(\gamma_1(a))\\\Im(\gamma_1(a))\end{pmatrix}
=\begin{pmatrix}a\\a\sqrt{3}\end{pmatrix}}\]
\[\vfunc{\vec{\upgamma}_2}{\bR}{\bR^2}{b}{\begin{pmatrix}\Re(\gamma_2(b))\\\Im(\gamma_2(b))\end{pmatrix}
=\begin{pmatrix}\cos b\\\sin(b)\sqrt{3}\end{pmatrix}}\]
Wir bauen die folgendes Gleichung System:
\[\begin{pmatrix}a\\a\sqrt{3}\end{pmatrix}=\begin{pmatrix}\cos(b)\\\sin(b)\sqrt{3}\end{pmatrix}\implies \cos(b)=\sin(b)\implies b_1=\frac{\pi}{4},\quad b_2=\pi+\frac{\pi}{4}\]
Wir haben 2 Schnittpunkten:
\newcommand{\schnt}[3]{P_{#1,#2,#3}}
\[\schnt{\gamma_1}{\gamma_2}{1}=\begin{pmatrix}\frac{\sqrt{2}}{2}\\\frac{\sqrt{2}}{2} \sqrt{3}\end{pmatrix},\quad \schnt{\gamma_1}{\gamma_2}{2}=-\schnt{\gamma_1}{\gamma_2}{1}\]
Wir berechnen die schnittwinkel zwischen $\gamma_1$ und $\gamma_2$, dafür linearisieren die Funktionen an der stelle $\schnt{\gamma_1}{\gamma_2}{1}$ und $\schnt{\gamma_1}{\gamma_2}{2}$:\\
\[\vec{\upgamma_1}'(a)=\begin{pmatrix}1\\\sqrt{3}\end{pmatrix}\]
\[\vec{\upgamma_2}'(b)=\begin{pmatrix}-\sin(b)\\\cos(b)\sqrt{3}\end{pmatrix}\]
für $\schnt{\gamma_1}{\gamma_2}{1}$:
\[\phi_1=\arccos(\frac{\begin{psmallmatrix}1\\\sqrt{3}\end{psmallmatrix}\cdot \begin{psmallmatrix}-\sin(b_1)\\\cos(b_1)\sqrt{3}\end{psmallmatrix}}{\left|\begin{psmallmatrix}1\\\sqrt{3}\end{psmallmatrix}\right|\left|\begin{psmallmatrix}-\sin(b_1)\\\cos(b_1)\sqrt{3}\end{psmallmatrix}\right|})
=\arccos(\frac{-\sin(b_1)+3\cos(b_1)}{\sqrt{1+3} \sqrt{3 \cos[2](b_1)+\sin[2](b_1)}})\]
\[=\arccos(\frac{-\frac{\sqrt{2}}{2}+3\frac{\sqrt{2}}{2}}{\sqrt{4} \sqrt{3 (\frac{\sqrt{2}}{2})^2+(\frac{\sqrt{2}}{2})^2}})=
\arccos(\frac{2\frac{\sqrt{2}}{2}}{2 \sqrt{3 \frac{1}{2}+\frac{1}{2}}})
=\arccos(\frac{\sqrt{2}}{2 \sqrt{4\frac{1}{2}}})
=\arccos(\frac{1}{2})=\frac{\pi}{3}\]
für $\schnt{\gamma_1}{\gamma_2}{2}$ folgt analog:
\[\phi_2=\arccos(\frac{-\sin(b_2)+3\cos(b_2)}{\sqrt{1+3} \sqrt{3 \cos[2](b_2)+\sin[2](b_2)}})
=\arccos(\frac{\frac{\sqrt{2}}{2}-3\frac{\sqrt{2}}{2}}{\sqrt{4} \sqrt{3 (\frac{\sqrt{2}}{2})^2+(\frac{\sqrt{2}}{2})^2}})
=\arccos(-\frac{1}{2})=\frac{2\pi}{3}\]

\item[b)]
Ermitteln Sie Schnittpunkt und Schnittwinkel der Bildkurven $f(\gamma_1)$ und $f(\gamma_2)$.\\
Die Schnittpunkt der Bildkurven von $f(\gamma_1)$ und $f(\gamma_2)$ ist gleich der Bild der schnittpunkt der Kurven, d.h

\end{itemize}


\end{section}
\begin{section}{Aufgabe}%Aufgabe 2
Ermitteln Sie fur die Möbius-Transformationen $h_1$ und $h_2$ mit:
\[h_1(z)=\frac{2z-1}{3z-2},\quad h_2(z)=\frac{z-1}{-2z+3}\]
die Komposition $h_1 \circ h_2$ und die inversen Transformationen $h_1^{-1}$ und $h_2^{-1}$.

\end{section}
\begin{section}{Aufgabe}%Aufgabe 3
Eine Möbius-Transformation $T$ werde durch die Angaben
\[f(1)=1,\quad f(\imag)=1+\imag,\quad f(-1+\imag)=2+2\imag\]
beschrieben.
\begin{itemize}
\item[a)]
Bestimmen Sie den Term $f(z)$ für $z \in \bC$.
\item[b)]
%\setmathfont{Latin Modern Math}

Bestimmen Sie von der offenen Halbgeraden $\gamma : t \mapsto t \imag, 0 < t < \infty$ rechnerisch
das Bild $f(\gamma)$ und beschreiben Sie es als geometrische Figur.
\end{itemize}
\end{section}
\end{document}
